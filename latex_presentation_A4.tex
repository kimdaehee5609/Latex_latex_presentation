%	-------------------------------------------------------------------------------
%
%
%
%
%
%
%
%
%	-------------------------------------------------------------------------------

%	\documentclass[10pt,xcolor=pdftex,dvipsnames,table]{beamer}
%	16:10
%	\documentclass[ aspectratio=1610, 10pt,blue,xcolor=pdftex,dvipsnames,table,handout]{beamer}
%	\documentclass[ aspectratio=1610, 10pt,blue,xcolor=pdftex,dvipsnames,table,handout,notes]{beamer}
%	16:9 
%	\documentclass[ aspectratio=169,  10pt,blue,xcolor=pdftex,dvipsnames,table,handout]{beamer}
%	\documentclass[ aspectratio=169,  10pt,blue,xcolor=pdftex,dvipsnames,table,handout,notes]{beamer}
%	14:9 
%	\documentclass[ aspectratio=149,  10pt,blue,xcolor=pdftex,dvipsnames,table,handout]{beamer}
	\documentclass[ aspectratio=149,  14pt,blue,xcolor=pdftex,dvipsnames,table,handout,notes]{beamer}

%	5:4
%	\documentclass[ aspectratio=54,   10pt,blue,xcolor=pdftex,dvipsnames,table,handout]{beamer}
%	4:3 default
%	\documentclass[ aspectratio=43, 10pt,blue,xcolor=pdftex,dvipsnames,table,handout]{beamer}
%	3:2 
% 	\documentclass[ aspectratio=32, 10pt,blue,xcolor=pdftex,dvipsnames,table,handout]{beamer}
		
		% Font Size
		%	default font size : 11 pt
		%	8,9,10,11,12,14,17,20
		%
		% 	put frame titles 
		% 		1) 	slideatop
		%		2) 	slide centered
		%
		%	navigation bar
		% 		1)	compress
		%		2)	uncompressed
		%
		%	Color
		%		1) blue
		%		2) red
		%		3) brown
		%		4) black and white	
		%
		%	Output
		%		1)  	[default]	
		%		2)	[handout]		for PDF handouts
		%		3) 	[trans]		for PDF transparency
		%		4)	[notes=hide/show/only]

		%	Text and Math Font
		% 		1)	[sans]
		% 		2)	[sefif]
		%		3) 	[mathsans]
		%		4)	[mathserif]


		%	---------------------------------------------------------	
		%	슬라이드 크기 설정 ( 128mm X 96mm )
		%	---------------------------------------------------------	
			\setbeamersize{text margin left=10mm}
			\setbeamersize{text margin right=10mm}

%			% Format presentation size to A4
%			\usepackage[size=a4]{beamerposter}		% A4용지 크기 사용

%			% Format presentation size to A4
%			\setlength{\paperwidth}{297mm}
%			\setlength{\paperheight}{210mm}
%			\setlength{\textwidth}{287mm}
%			\setlength{\textheight}{200mm}   

%			% Format presentation size to A4 길게
			\setlength{\paperwidth}	{210mm}
			\setlength{\paperheight}	{297mm}
			\setlength{\textwidth}		{190mm}
			\setlength{\textheight}	{287mm}   


	%	========================================================== 	Package
		\usepackage{kotex}						% 한글 사용
		\usepackage{amssymb,amsfonts,amsmath}	% 수학 수식 사용
		\usepackage{color}						%
		\usepackage{colortbl}					%



		%	---------------------------------------------------------	
		%		특수문자 입력용
		%	---------------------------------------------------------	
			\usepackage{textcomp}
			\usepackage{gensymb}
			\usepackage{marvosym} 	% 구조계산 그림용


		%	---------------------------------------------------------	
		%		Table
		%	---------------------------------------------------------	
			\usepackage{booktabs}			% toprule cmidrule midrule bottomrule  
			\usepackage{longtable}			%
			\usepackage{tabularx}			%
			\usepackage{array}				%
			\usepackage{bigstrut}			%


		%	---------------------------------------------------------	
		%		Box
		%	---------------------------------------------------------	
			\usepackage{adjustbox}

		
		%	---------------------------------------------------------	
		%		유인물 출력 : 출력할대 조정해서 출력 할것
		%	---------------------------------------------------------	

			\usepackage{pgfpages}
%			\pgfpagesuselayout{2 on 1}[letterpaper]
%			\pgfpagesuselayout{4 on 1}[letterpaper]
%			\pgfpagesuselayout{8 on 1}[letterpaper]

%			\pgfpagesuselayout{resize to}[a4paper,landscape,border shrink=5mm]
			\pgfpagesuselayout{resize to}[a4paper,border shrink=5mm]
%			\pgfpagesuselayout{2 on 1}[a4paper,border shrink=5mm]
%			----------------------------------------------------- 출력 시 설정	1
%			\pgfpagesuselayout{2 on 1}[a4paper]
%			\usecolortheme{seagull}	% 휜색
%			----------------------------------------------------- 출력 시 설정	2
%			\pgfpagesuselayout{2 on 1}[a4paper,border shrink=5mm]%
%			\usecolortheme{dove}
%			----------------------------------------------------- 
%			\pgfpagesuselayout{4 on 1}[a4paper,border shrink=5mm]
%			\pgfpagesuselayout{8 on 1}[a4paper,border shrink=5mm]

			\usepackage{handoutWithNotes}
%			\pgfpagesuselayout{1 on 1 with notes}[a4paper,border shrink=5mm]
%			\pgfpagesuselayout{2 on 1 with notes}[a4paper,border shrink=5mm]
%			\pgfpagesuselayout{3 on 1 with notes}[a4paper,border shrink=5mm]
%			\pgfpagesuselayout{4 on 1 with notes}[a4paper,border shrink=5mm]

%			\pgfpagesuselayout{2 on 1}[letterpaper]
%			\pgfpagesuselayout{2 on 1}[letterpaper]
%			\pgfpagesuselayout{2 on 1}[letterpaper]



	%		========================================================= 	Theme

		%	---------------------------------------------------------	
		%	전체 테마
		%	---------------------------------------------------------	
		%	테마 명명의 관례 : 도시 이름
%			\usetheme{default}			%
%			\usetheme{Madrid}    		%
%			\usetheme{CambridgeUS}    	% -red, no navigation bar
			\usetheme{Antibes}			% -blueish, tree-like navigation bar

		%	----------------- table of contents in sidebar
%			\usetheme{Berkeley}		% -blueish, table of contents in sidebar
									% 개인적으로 마음에 듬

%			\usetheme{Marburg}			% - sidebar on the right
%			\usetheme{Hannover}		% 왼쪽에 마크
%			\usetheme{Berlin}			% - navigation bar in the headline
%			\usetheme{Szeged}			% - navigation bar in the headline, horizontal lines
%			\usetheme{Malmoe}			% - section/subsection in the headline

%			\usetheme{Singapore}
%			\usetheme{Amsterdam}

		%	---------------------------------------------------------	
		%	색 테마
		%	---------------------------------------------------------	
%			\usecolortheme{albatross}	% 바탕 파란
%			\usecolortheme{crane}		% 바탕 흰색
%			\usecolortheme{beetle}		% 바탕 회색
%			\usecolortheme{dove}		% 전체적으로 흰색
%			\usecolortheme{fly}		% 전체적으로 회색
%			\usecolortheme{seagull}	% 휜색
%			\usecolortheme{wolverine}	& 제목이 노란색
%			\usecolortheme{beaver}

		%	---------------------------------------------------------	
		%	Inner Color Theme 			내부 색 테마 ( 블록의 색 )
		%	---------------------------------------------------------	

%			\usecolortheme{rose}		% 흰색
%			\usecolortheme{lily}		% 색 안 칠한다
%			\usecolortheme{orchid} 	% 진하게

		%	---------------------------------------------------------	
		%	Outter Color Theme 		외부 색 테마 ( 머리말, 고리말, 사이드바 )
		%	---------------------------------------------------------	

%			\usecolortheme{whale}		% 진하다
%			\usecolortheme{dolphin}	% 중간
%			\usecolortheme{seahorse}	% 연하다

		%	---------------------------------------------------------	
		%	Font Theme 				폰트 테마
		%	---------------------------------------------------------	
%			\usfonttheme{default}		
			\usefonttheme{serif}			
%			\usefonttheme{structurebold}			
%			\usefonttheme{structureitalicserif}			
%			\usefonttheme{structuresmallcapsserif}			



		%	---------------------------------------------------------	
		%	Inner Theme 				
		%	---------------------------------------------------------	

%			\useinnertheme{default}
			\useinnertheme{circles}		% 원문자			
%			\useinnertheme{rectangles}		% 사각문자			
%			\useinnertheme{rounded}		% 깨어짐
%			\useinnertheme{inmargin}			




		%	---------------------------------------------------------	
		%	이동 단추 삭제
		%	---------------------------------------------------------	
%			\setbeamertemplate{navigation symbols}{}

		%	---------------------------------------------------------	
		%	문서 정보 표시 꼬리말 적용
		%	---------------------------------------------------------	
%			\useoutertheme{infolines}


			
	%	---------------------------------------------------------- 	배경이미지 지정
%			\pgfdeclareimage[width=\paperwidth,height=\paperheight]{bgimage}{./fig/Chrysanthemum.jpg}
%			\setbeamertemplate{background canvas}{\pgfuseimage{bgimage}}

		%	---------------------------------------------------------	
		% 	본문 글꼴색 지정
		%	---------------------------------------------------------	
%			\setbeamercolor{normal text}{fg=purple}
%			\setbeamercolor{normal text}{fg=red!80}	% 숫자는 투명도 표시


		%	---------------------------------------------------------	
		%	itemize 모양 설정
		%	---------------------------------------------------------	
%			\setbeamertemplate{items}[ball]
%			\setbeamertemplate{items}[circle]
%			\setbeamertemplate{items}[rectangle]


		%	---------------------------------------------------------	
		%	상자 모양새 설정
		%	---------------------------------------------------------	
%			\setbeamertemplate{blocks}[rounded,shadow=true]
%			\begin{block}
%			\begin{theorem}
%			\begin{lemma}
%			\begin{proof}
%			\begin{corollary}
%			\begin{example}
%			\begin{exampleblock}
%			\begin{alertblock}




		\setbeamercovered{dynamic}






% ------------------------------------------------------------------------------
% Begin document (Content goes below)
% ------------------------------------------------------------------------------
	\begin{document}
	

			\title{LATEX}
			\subtitle{사용설명서}
			\author{김대희}
			\date[2015.06.30]{2015년 7월}
			\institute[KTS]{(주)서영엔지니어링 \url{http://symsone.seoyeong.co.kr/}}



	%	==========================================================
	%
	%	----------------------------------------------------------
		\begin{frame}[plain]
		\titlepage
		\end{frame}



	%	==========================================================
	%		자주 사용하는 기능들
	%	----------------------------------------------------------



	%	----------------------------------------------------------
	%		미리 정의된 문자열
	%	----------------------------------------------------------
		\begin{frame}[t,allowframebreaks]{미리 정의된 문자열}

			\begin{center}
			\rowcolors{1}{blue!10}{yellow!10}
			\begin{table}
			\begin{tabular}{ l l l  }
				명령어	&사용예	&설명\\
				\hline
				\textbackslash today 	&\today 		&현재 사용 언어에서의 현재 날짜 표기\\
				\textbackslash TeX 	&\TeX 		&최고의 조판 시스템의 이름\\
				\textbackslash LaTeX 	&\LaTeX 		&지금 우리가 배우고 있는 것의 이름\\
				\textbackslash LaTeXe 	&\LaTeXe		&LATEX의 최신판\\
			\end{tabular}
			\end{table}
			\end{center}


	%	----------------------------------------------------------
			\begin{columns}[t]
			\begin{column}{.4\textwidth}
			\begin{block} {특수문자}
			\begin{enumerate}
			\item \# : \textbackslash \# \\
			\item \$ : \textbackslash \$ \\
			\item \% : \textbackslash \% \\
			\item \& : \textbackslash \& \\
			\item \_ : \textbackslash \_ \\
			\item \{ : \textbackslash \{ \\
			\item \} : \textbackslash \} \\
			\item \textbackslash  : \textbackslash textbackslash
			\end{enumerate}
			\end{block}
			\end{column}

			\begin{column}{.4\textwidth}
			\begin{block} {특수문자}
			\begin{enumerate}
			\item \_ : \textbackslash \_ \\
			\item \^{} : \textbackslash \^{} \\
			\item $\sim$ : \$ \textbackslash sim \$
			\end{enumerate}
			\end{block}
			\end{column}

			\end{columns}

	%	----------------------------------------------------------

			\begin{columns}[t]
			\begin{column}{.4\textwidth}
			\begin{block} {특수문자 : 수학기호로 처리}
			\begin{enumerate}
			\item $\cdot$ 		: \$ $\cdot$ \$ \\
			\item $\circ$ 		: \$ $\circ$ \$ \\
			\item $\bullet$ 		: \$ $\bullet$ \$ \\
			\item $\centerdot$		: \$ $\centerdot$ \$ \\
			\item $\div$ 			: \$ $\div$ \$ \\
			\item $\times$ 		: \$ $\times$ \$ \\
			\end{enumerate}
			\end{block}
			\end{column}

			\begin{column}{.4\textwidth}
			\begin{block} {특수문자}
%			\begin{enumerate}
%			\item \approx		:  	\\
%			\item \neq		: 	\\
%			\item \infty		: 	\\
%			\end{enumerate}
			\end{block}
			\end{column}
			\end{columns}


			\textcircled{\small}\\


		\end{frame}


	%	----------------------------------------------------------
		\begin{frame}[t]{특수문자 : textcomp, gensymb package 사용 }

			\begin{columns}[t]
			\begin{column}{.4\textwidth}
					\begin{block} { usepackage textcomp }
					\begin{enumerate}
					\item[\textcelsius 		]	\textbackslash textcelsius
					\item[\textpertenthousand	]	\textbackslash textpertenthousand
					\item[\textperthousand		]	\textbackslash textperthousand
					\item[\textreferencemark	]	\textbackslash textreferencemark
					\end{enumerate}
					\end{block}
			\end{column}

			\begin{column}{.4\textwidth}
					\begin{block} { usepackage gensymb}
					\end{block}

			\end{column}
			\end{columns}

		\end{frame}


	%	----------------------------------------------------------
		\begin{frame}[t,shrink=0]{특수문자 : 구조 계산 그림용 package marvosym}

			\begin{columns}[t]
			\begin{column}{.4\textwidth}
			\begin{block} { 모델링 }
			\begin{enumerate}
			\item[\Beam			]	\textbackslash Beam	
			\item[\Force			]	\textbackslash Force	
			\item[\Lineload		]	\textbackslash Lineload	
			\item[\Lefttorque		]	\textbackslash Lefttorque	
			\item[\Righttorque		]	\textbackslash Righttorque	
			\item[\Fixedbearing	]	\textbackslash Fixedbearing	
			\item[\Bearing		]	\textbackslash Bearing
			\item[\Loosebearing	]	\textbackslash Loosebearing	

			\item[\Rectpipe		]	\textbackslash Rectpipe
			\item[\Squarepipe		]	\textbackslash Squarepipe	
			\item[\Circpipe		]	\textbackslash Circpipe	
			\end{enumerate}
			\end{block}
			\end{column}

			\begin{column}{.4\textwidth}
			\begin{block} { 단면 }
			\begin{enumerate}
			\item[\Circsteel		]	\textbackslash Circsteel	
			\item[\Octosteel		]	\textbackslash Octosteel	
			\item[\Hexasteel		]	\textbackslash Hexasteel
			\item[\Squaresteel•		]	\textbackslash Squaresteel•	
			\item[\Rectsteel		]	\textbackslash Rectsteel	

			\item[\Tsteel			]	\textbackslash Tsteel	
			\item[\RoundedTsteel	]	\textbackslash RoundedTsteel	
			\item[\TTsteel		]	\textbackslash TTsteel	
			\item[\RoundedTTsteel	]	\textbackslash RoundedTTsteel
			\item[\Flatsteel		]	\textbackslash Flatsteel	
			\item[\Lsteel			]	\textbackslash Lsteel	
			\item[\RoundedLsteel	]	\textbackslash RoundedLsteel	
			\end{enumerate}
			\end{block}
			\end{column}
			\end{columns}



			\begin{block}{structural analysis 사용준비}
			\begin{itemize}
			\item[] \textbackslash usepackage\{structuralanalysis\}
			\item[] \textbackslash usepackage\{3dstructuralanalysis\}
			\item[] 
			\item[] \textbackslash begin\{tikzpicture\}
			\item[] \textbackslash point\{a\}\{0\}\{0\};
			\item[] \textbackslash beam\{2\}\{a\}\{b\}[0][1];
			\item[] \textbackslash support\{1\}\{a\}[0];
			\item[] \textbackslash end\{tikzpicture\}
			\end{itemize}
			\end{block}




		\end{frame}




	%	----------------------------------------------------------
	%		치수
	%	----------------------------------------------------------
		\begin{frame}[t]{치수}

			\begin{block} {치수}
			\begin{description}[12345678901234567890]
			\item [\textbackslash text width ]
			\item [\textbackslash text height ]
			\item [\textbackslash line width ]
			\item [\textbackslash paper width ]
			\item [\textbackslash paper height ]
			\end{description}
			\end{block}


		\end{frame}




	%	----------------------------------------------------------
	%		선 그리기
	%	----------------------------------------------------------
		\begin{frame}[t]{선 그리기}

			\begin{block} {선 그리기}
			\textbackslash rule \{ \textbackslash linewidth \} \{ 두께 \}
			\end{block}

			\begin{example}
			\textbackslash rule \{ 4cm \} \{ 2mm \}
			\end{example}

			\rule{4cm}{2mm}

		\end{frame}




	%	==========================================================
	%		글꼴 모양
	%	----------------------------------------------------------



	%	----------------------------------------------------------
	%		글꼴 모양 과 크기
	%	----------------------------------------------------------


		\begin{frame}[t]{글꼴 모양과 크기}

			\begin{columns}[t]
			\begin{column}{.5\textwidth}
					\begin{block}{글꼴 모양}
					\begin{table}
					\begin{tabular}{ l l l  }
					명령어	&환경	&결과\\
					\hline
					\textbackslash textnarmal 	&textnarmal 	&\textnormal{결과}\\
					\textbackslash textit 		&itshape		&\textit{결과}\\
					\textbackslash emph 		&없음		&\emph{결과}\\
					\textbackslash textbf 		&bfseries		&\textbf{결과}\\
					\textbackslash underline 	&밑줄		&\underline{결과}\\
					\end{tabular}
					\end{table}
					\end{block}
			\end{column}
			\begin{column}{.5\textwidth}
					\begin{block}{글꼴 크기}
					\begin{tabular}{ l l l  }
					명령어		&결과\\
					\hline
					\textbackslash tiny			&\tiny{결과}\\
					\textbackslash scriptsize		&\scriptsize{결과}\\
					\textbackslash footnotesize		&\footnotesize{결과}\\
					\textbackslash normalsize		&\normalsize{결과}\\
					\textbackslash large			&\large{결과}\\
					\textbackslash Large			&\Large{결과}\\
					\textbackslash LARGE			&\LARGE{결과}\\
					\textbackslash huge			&\huge{결과}\\
					\textbackslash Huge			&\Huge{결과}\\
					\end{tabular}
					\end{block}
			\end{column}
			\end{columns}


		\end{frame}






	%	==========================================================
	%		문서 클래스
	%	----------------------------------------------------------


	%	----------------------------------------------------------
	%		문서 클래스
	%	----------------------------------------------------------
		\begin{frame}[t,shrink=00]{문서 클래스}


			\begin{block} {장, 절의 설정}
			\begin{description}[1234567890]
			\item [article] 과학 학술지, 프리젠테이션, 짧은 보고서, 프로그램 문서, 초대장 등에 쓰이는 아티클용 클래스
			\item [proc article] 클래스에 기초한 프로시딩을 위한 클래스 
			\item [minimal] 최소 문서 양식 클래스. 페이지 크기와 기본 글꼴만을 설정한다. 주로 디버깅을 위하여 사용함.
			\item [report] 여러 장(chapter)으로 이루어진 긴 보고서, 작은 책, 박사학위 논문 등에 쓰이는 클래스.
			\item [book] 진짜 책을 만들기 위한 클래스.
			\item [slides] 슬라이드 제작용 클래스. 
			\end{description}
			\end{block}

	%	----------------------------------------------------------

			\begin{block} {문서 클래스 옵션}
			\begin{description}[1234567890]
			\item [10pt] 11pt, 12pt 문서 기본 글꼴 크기를 설정한다.
			\item [letterpaper] a4paper,a5paper, b5paper, executivepaper,legalpaper
			\item [fleqn] 수식을 가운데 정렬이 아닌 왼쪽 정렬로 식자한다.
			\item [leqno] 수식 번호를 수식의 오른쪽이 아닌 왼쪽에 표시되도록 한다.
			\item [titlepage] notitlepage 표지 뒤에 새로운 페이지를 시작하도록 할 것인지 지정한다. report와 book은 새 페이지를 만든다.
			\item [onecolumn] twocolumn 문서를 1단 또는 2단으로 조판하도록 지시한다.
			\item [twoside] oneside 양면인쇄용 출력물 생성.\\ 단면(article,report) 양면 (book)
			\item [landscape] 레이아웃을 가로가 긴 형식(landscape)으로 변경한다.
			\item [openright] openany 새로운 장을 홀수쪽에서 시작.\\
								book 클래스에서는 홀수쪽에서 시작하는 것이 기본값이다.
			\end{description}
			\end{block}

		\end{frame}












	%	==========================================================
	%		패키지
	%	----------------------------------------------------------
		\begin{frame}[t]{패키지}


			\begin{block} {표 작성 관련 패키지}
			\begin{description}[12345678901234567890]
			\item[booktabs]	toprule cmidrule midrule bottomrule  
			\item[longtable]	
			\item[tabularx]	
			\item[array]		
			\item[bigstrut]	
			\end{description}
			\end{block}


			\begin{block} {특수문자 입력용}
			\begin{description}[12345678901234567890]
			\item[textcomp]
			\item[gensymb]
			\item[marvosym] 	 구조계산 그림용
			\end{description}
			\end{block}


			\begin{block} {한번씩 사용하는 기능들}
			\begin{description}[12345678901234567890]
			\item[blindtext] 임의의 문서를 자동 생성한다. 코딩 시 테스트용으로 주로 사용
			\end{description}
			\end{block}

			\begin{block} {Page 관련 패키지}
			\begin{description}[12345678901234567890]
			\item[afterpage]	 다음페이지가 나온면 어떻게 하라는 명령 정의 패키지
			\item[fullpage]			
			\item[pdflscape]	 페이지를 전체 돌린다.	
			\item[lscape]		그림이나 표등 부분을 돌린다
			\end{description}
			\end{block}

		\end{frame}



	%	==========================================================
	%		쪽 양식
	%	----------------------------------------------------------
		\begin{frame}[t]{쪽양식}

			\begin{block} {latex에서 미리 정의된 쪽양식}
			\begin{description}[1234567890]
			\item [\textbf{plain}] 쪽 번호를 쪽의 아래쪽 바닥글에 중앙정렬하여 찍는다.\\
								쪽 양식의 기본값이다.
			\item [\textbf{headings}] 현재 장 표제와 쪽 번호를 각 쪽의 머리글에 적는다.\\
								바닥글은 비운다.
			\item [\textbf{empty}] 머리글과 바닥글을 모두 비운다.
			\end{description}
			\end{block}


	%	---------------------------------------------------------- 페이지 칼라
			\begin{block} {페이지 칼라 변경}
			\begin{enumerate}
			\item	\textbackslash usepackage \{xcolor\} 
			\item	\textbackslash usepackage \{afterpage\}
			\item	
			\item	\textbackslash pagecolor \{black\} \textbackslash  afterpage\{ \textbackslash  nopagecolor\{\}\}
			\item	\textbackslash color\{white\} 글자색 흰색으로
			\end{enumerate}
			\end{block}


		\end{frame}


	%	==========================================================
	%		장, 절의 설정
	%	----------------------------------------------------------
		\begin{frame}[t]{장, 절의 설정}

			\begin{block} {장, 절의 설정}
			\begin{description}[1234567890]
			\item[부]		\textbackslash part (-1)
			\item[장]		\textbackslash chapter (0)
			\item[절]		\textbackslash section	(1)
			\item[소절]	\textbackslash sub section	(2)
			\item[소소절]	\textbackslash sub sub section (3)
			\item[문단]	\textbackslash paragraph (4)
			\item[소문단]	\textbackslash sub paragraph (5)
			\end{description}
			\end{block}

		\textbackslash setcounter \{ sec num depth \} \{n\} 명령에 의해 장 번호의 설치 깊이를 설정\\
		n을 2로 설정하면 subsection 까지 번호 부여
		\end{frame}




	%	==========================================================
	%		title page
	%	----------------------------------------------------------
		\begin{frame}[t,shrink=10]{표지 작성}


			\begin{columns}[t]
			\begin{column}{.4\textwidth}
					\begin{block} {표지작성}
					\begin{description}[1234567890]
					\item [\textbackslash title] 	\{문서 제목\}
					\item [\textbackslash author]	\{문서 저자\}
					\item [\textbackslash date] 	\{작성일\}
					\item []
					\item [\textbackslash maketitle] 타이틀 표시
					\end{description}
					\end{block}
			\end{column}

			\begin{column}{.6\textwidth}
					\begin{block} {사용자 정의 : \textbackslash begin\{titlepage\} \textbackslash end \{titlepage\} }
					\begin{itemize}
					\item []\textbackslash begin\{titlepage\}
					\item []\textbackslash begin\{center\}
					\item []\textbackslash vspace*\{1cm\}  윗부분 여백지정
					\item []\textbackslash Huge \textbackslash textbf\{보고서 제목\} \textbackslash \textbackslash 
					\item []\textbackslash vspace\{0.5cm\}
					\item []\textbackslash Large 보고서 부제목 \textbackslash \textbackslash 
					\item []\textbackslash vfill
					\item []\textbackslash Date
					\item []\textbackslash vfill
					\item []\textbackslash textbf\{지은이\}\textbackslash \textbackslash 
					\item []\textbackslash vspace\{1.5cm\}
					\item []\textbackslash Large Department Name\textbackslash \textbackslash 
					\item []\textbackslash vspace\{2.0cm\}  아래부분 여백 정의
					\item []\textbackslash end \{center\}
					\item []\textbackslash end \{titlepage\}
					\end{itemize}
					\end{block}
			\end{column}
			\end{columns}



		표지의 표시 순서 변경 방법
		\end{frame}

%				%	----------------------------------------------------
%				%	문서 표지 페이지
%				%	----------------------------------------------------
%					\begin{titlepage}
%					\begin{center}
%					\vspace*{1cm}  %	----------------- 윗부분 여백지정
%					\rule{1.0\textwidth} {0.1mm}
%					\Huge \textbf{보고서 제목} 
%					\par
%					\rule{1.0\textwidth} {0.1mm}
%					\vspace{0.5cm}
%					\Large 보고서 부제목 
%					\par
%					\vfill
%					\Date{\today}
%					\vfill
%					\textbf{지은이}
%					\par
%					\vspace{1.5cm}
%					\Large Department Name
%					\par
%					\vspace{2.0cm}  %	----------------- 아래부분 여백 정의
%					\end{center}
%					\end{titlepage}
%				%	----------------------------------------------------














	%	==========================================================
	%		초록 작성
	%	----------------------------------------------------------
		\begin{frame}[t]{초록 작성}

			\begin{block} {초록 작성}
			\begin{description}[12345678901234567]
			\item [\textbackslash begin\{abstract\}] 	문서 제목
			\item [초록 내용]						초록 내용
			\item [\textbackslash end\{abstract\}] 	작성일
			\end{description}
			\end{block}


		\end{frame}




	%	==========================================================
	%		목차 작성
	%	----------------------------------------------------------
		\begin{frame}[t]{목차 작성}

			\begin{block} {목차 작성}
			\begin{description}[12345678901234567890]
			\item [\textbf{\textbackslash table of contents}] 	문서 내용 목차
			\item [\textbf{\textbackslash list of figures}]		그림 목차
			\item [\textbf{\textbackslash list of tables}] 		표 목차
			\end{description}
			\end{block}

			\begin{block} {목차 표지 범위 설정}
			\textbackslash setcounter \{ tocdepth \} 
			\end{block}

			\begin{block} {목차에서 페이지 나누기}
			\textbackslash addtocontents \{ toc \} \{ \textbackslash protect \textbackslash newpage \}
			\end{block}

			\begin{block} {번호 설정 깊이 }
			\textbackslash setcounter \{ secnumdepth \} \{ n \} 
			\end{block}

			\begin{tabular}{ l l }
				\hline
				부(part)				&-1\\
				장(chapter)			&0\\
				절(section)			&1\\
				소절(subsection)		&2\\
				소소절(subsubsection)	&3\\
				문단(paragraph)		&4\\
				소문단(subparagraph)	&5\\
				\hline
			\end{tabular}

		\end{frame}



	%	==========================================================
	%		주석문 처리
	%	----------------------------------------------------------

		\begin{frame}[t]{주석문 처리}

			\begin{block} {주석문 처리}
			\textbackslash \%
			\end{block}

		\end{frame}


	%	----------------------------------------------------------
	%		각주, 난외주
	%	----------------------------------------------------------

		\begin{frame}[t]{각주, 난외주}

			\begin{block} {각주}
			\textbackslash footnote \{ 각주 내용 \}
			\end{block}

			\begin{block} {난외주}
			\textbackslash marginpar \{ 난외주 내용 \}
			\end{block}


		\end{frame}



	%	----------------------------------------------------------
	%		인용문
	%	----------------------------------------------------------

		\begin{frame}[t]{인용문}

			\begin{block} {인용문}
			\begin{itemize}
			\item[]	\textbackslash begin \{ quote \} ~인용문 내용 \textbackslash end \{ quote \} \\
			\item[]	\textbackslash begin \{ quotation \} ~인용문 내용 \textbackslash end \{ quotation \} 
			\end{itemize}
			\end{block}

			\begin{example}
				\begin{quote} 인용문 내용 \end{quote} 
				\begin{quotation} 인용문 내용 \end{quotation}
			\end{example}


		\end{frame}





	%	----------------------------------------------------------
	%		 상호 참조
	%	----------------------------------------------------------

		\begin{frame}[t]{상호참조}

			\begin{block} {상호참조}
				\begin{itemize}
				\item \textbackslash label \{참조 기호 \}
				\item \textbackslash ref \{참조할 기호 \}
				\item \textbackslash pageref \{참조할 기호 \}
				\end{itemize}
			\end{block}

		\end{frame}


	%	----------------------------------------------------------
	%		하이퍼링크
	%	----------------------------------------------------------

		\begin{frame}[t]{하이퍼링크}

			\begin{block} {하이퍼링크}
				\begin{itemize}
				\item [] \textbackslash  url\{하이퍼링크\}																\end{itemize}
			\end{block}

			\begin{example}
				\begin{itemize}
				\item \textbackslash url\{http://www.band.us/\#/band/53125310\} 
				\item \textbackslash url\{http://www.seoyeong.co.kr/\}
				\item \textbackslash url\{http://symsone.seoyeong.co.kr\}
				\item \textbackslash url\{http://syerp.seoyeong.co.kr/\}						
				\end{itemize}
			\end{example}

				\url{http://www.seoyeong.co.kr/}

		\end{frame}














	%	==========================================================
	%		Chapter styles : FncyChap package
	%	----------------------------------------------------------
		\begin{frame}[t]{Chapter styles : FncyChap package}

%		\setbeamercolor{block title}	{use=structure,fg=white,bg=purple!75!black}
		\setbeamercolor{block title}	{use=structure,fg=black,bg=white!20!white}
		\setbeamercolor{block body}		{use=structure,fg=black,bg=white!20!white}


		\begin{block}{\textbf{1. 사용준비}}
			\textbackslash usepackage [style]\{fncychap\}
		\end{block}


		\begin{block}{\textbf{2. style 종류}}
			\begin{enumerate}
			\item Sonny
			\item Lenny
			\item Glenn
			\item Conny
			\item Rejne
			\item Bjarne
			\item Bjornstrup
			\end{enumerate}
		\end{block}


		\begin{block}{\textbf{3. style 모양 제어}}
			\begin{itemize}
			\item \textbackslash makeatletter		   
			\item \textbackslash ChNameVar{\large\bf}
			\item \textbackslash ChTitleVar{\large\bf}
			\item \textbackslash makeatother
			\end{itemize}
		\end{block}


		\end{frame}

















	%	==========================================================
	%		Page Size
	%	----------------------------------------------------------
		\begin{frame}[c]{Page size}


			\begin{block} {Page size}
			\begin{description}[12345678901234567890]
			\item	[a4paper] 210mm $\times$ 297mm
			\item	[a5paper]
			\item	[b5paper]
			\item	[letterpaper]
			\item	[legalpaper]
			\item	[executivepaper]
			\end{description}
			\end{block}


			\begin{example}
			\begin{itemize}
			\item[]	\textbackslash documentclass[a4paper,landscape,12pt]\{article\}
			\item[]	\textbackslash documentclass[a5paper,landscape,12pt]\{article\}
			\end{itemize}
			\end{example}

			\begin{itemize}
			\item[]	landscape
			\item[]	portrait
			\end{itemize}


		\end{frame}

	%	----------------------------------------------------------
	%		Page Size : geometry package ( 전체 여백을 지정한다. )
	%	----------------------------------------------------------
		\begin{frame}[t]
		\frametitle{geometry package}

			\begin{block} {geometry package}
			\textbackslash usepackage
						[ top=0.0mm, \\
						\hspace{6em} left=0.0mm, \\
						\hspace{6em} bottom=0.0mm, \\
						\hspace{6em} righr=0.0mm] \\
						\hspace{6em} \{geometry\}
			\end{block}

	%	----------------------------------------------------------

			\begin{block} {geometry 옵션 인자}
			\begin{description}[12345678901234567890]
			\item	[paperwidth]		=25cm
			\item	[paperheight]		=35cm
			\item	[papersize]		=\{25cm,35cm\}
			\item	[width]			=20cm \% total width
			\item	[heigth]			=30cm \% total heigth
			\item	[total]			=\{20cm,30cm\}
			\item	[textwidth]		=18cm \% width - marginpar
			\item	[textheight]		=25cm \% heigth - header - footer
			\item	[body]			=\{18cm,25cm\}
			\item	[left]			=3cm \% left margin
			\item	[right]			=1.5cm \% right margin
			\item	[hmargin]			=\{3cm,2cm\}
			\item	[top]			=2cm \% top margin
			\item	[bottom]			=3cm \% bottom margin
			\item	[vmargin]			=\{2cm,3cm\}
			\item	[marginparwidth]	=2cm
			\item	[head]			=1cm \% header space
			\end{description}
			\end{block}

			\begin{block} {Page Size}
			\end{block}

			\begin{block} {layout Size}
			\end{block}

			\begin{block} {Body Size}
			\end{block}

			\begin{block} {Margin Size}
			\end{block}

		\end{frame}


	%	----------------------------------------------------------
	%		Page layout : 페이지 레이아웃을 바꾼다
	%	----------------------------------------------------------
		\begin{frame}[t]
		\frametitle{geometry package}

			\begin{block} {page layout 변경}
			\begin{itemize}
			\item[] \textbackslash new geometry \{ \}
			\item[] \textbackslash re store geometry \{ \}
			\item[] 
			\item[] \textbackslash save geometry \{ \(<\) name \(>\)  \}
			\item[] \textbackslash re store geometry \{ \(<\) name \(>\)  \}
			\end{itemize}

			\end{block}

	%	----------------------------------------------------------

			\begin{example} {geometry 옵션 인자}
			\begin{itemize}
			\item[] \textbackslash usepackage[hmargin=3cm]\{geometry\}
			\item[] 
			\item[] \textbackslash begin\{document\}
			\item[] \hspace{4em}	Layout L1
			\item[] \textbackslash newgeometry\{left=3cm,right=1cm,bottom=0.1cm\}
			\item[] \hspace{4em}	Layout L2 (new)
			\item[] \textbackslash restoregeometry
			\item[] \hspace{4em}	Layout L1 (restored)
			\item[] \textbackslash newgeometry\{margin=1cm,includefoot\}
			\item[] \textbackslash savegeometry\{L3\}
			\item[] \hspace{4em}	L3 (new, saved)
			\item[] \textbackslash restoregeometry
			\item[] \hspace{4em}	L1 (restored)
			\item[] \textbackslash newgeometry\{margin=1cm,includefoot\}
			\item[] \hspace{4em}	L4 (new)
			\item[] \textbackslash loadgeometry\{L3\}
			\item[] \hspace{4em}	L3 (loaded)
			\item[] \textbackslash end\{document\}
			\end{itemize}
			\end{example}


		\end{frame}

	%	----------------------------------------------------------
	%		Page rotation : 페이지를 돌린다
	%	----------------------------------------------------------
		\begin{frame}[t]
		\frametitle{페이지를 돌린다}

			\begin{block} {페이지 전체를 돌리기}
			\begin{itemize}
			\item[] \textbackslash usepackage \{ pdflscape \}
			\item[] 
			\item[] \textbackslash begin \{ landscape \}
			\item[] \textbackslash end \{ landscape \}
			\end{itemize}
			\end{block}

			\begin{block} {페이지는 놓아두고 그림이나 표등을 돌린다}
			\begin{itemize}
			\item[] \textbackslash usepackage \{ lscape \}
			\item[] 
			\item[] \textbackslash begin \{ landscape \}
			\item[] \textbackslash end \{ landscape \}
			\end{itemize}
			\end{block}


		\end{frame}


	%	==========================================================
	%		문단 모양
	%	----------------------------------------------------------


	%	----------------------------------------------------------
	%		Line and page Breaking
	%	----------------------------------------------------------
		\begin{frame}[t]{페이지 나누기 줄바꾸기 Line and page Breaking}

			\begin{block} {Line and page Breaking}
			\begin{enumerate}
			\item	\textbackslash \textbackslash 
			\item	\textbackslash \textbackslash  *
			\item	\textbackslash clear page
			\item	\textbackslash clear double page
			\item	\textbackslash hyphenation
			\item	\textbackslash line break
			\item	\textbackslash new line
			\item	\textbackslash no line break
			\item	\textbackslash no page break
			\item	\textbackslash page break
			\end{enumerate}
			\end{block}

		\end{frame}

	%	----------------------------------------------------------
	%		문단 첫줄 들여쓰기
	%	----------------------------------------------------------
		\begin{frame}[t]{문단 첫줄 들여쓰기}

			\begin{block} {문단 첫줄 들여쓰기}
			\textbackslash par indent
			\end{block}

			\begin{example}
			\textbackslash setlength \{ \textbackslash parindent \} \{ 0.0cm \}\\
			\end{example}

		
		\end{frame}

	%	----------------------------------------------------------
	%		문단과 문단 사이의 간격
	%	----------------------------------------------------------
		\begin{frame}[t]{문단과 문단 사이의 간격}

			\begin{block} {문단과 문단 사이의 간격}
			\textbackslash par skip
			\end{block}

			\begin{example}
			\textbackslash setlength \{ \textbackslash parskip \} \{ 0.0pt \}\\
			\textbackslash setlength \{\textbackslash parskip \} \{ 1cm plus 4mm minus\}	\\		
			\end{example}

			\begin{example}
			\textbackslash small skip
			\textbackslash med skip
			\textbackslash big skip
			\end{example}

		
		\end{frame}


	%	----------------------------------------------------------
	%		문단내에서의 줄간격
	%	----------------------------------------------------------
		\begin{frame}[t]{문단내에서의 줄간격}

			\begin{block} {문단내에서의 줄간격}
			\textbackslash usepackage \{ setspace \} \\
			\begin{itemize}
			\item[]	\textbackslash singlespacing 
			\item[]	\textbackslash onehalfspacing 
			\item[]	\textbackslash doublespacing 
			\item[]	\textbackslash setstretch \{ $<$ $>$ \} 
			\item[]	\textbackslash linespread \{ $<$ factor $>$ \}
			\end{itemize}

			\end{block}

			\begin{example}
			\textbackslash linespread \{ 1.6 \} : double-spacing \\
			\textbackslash linespread \{ 1.3 \} : one-and-a half spacing 
			\end{example}
	
			\begin{example}
			\textbackslash begin \{ doublespace \} \\
			.............. \\
			\textbackslash end \{ doublespace \} \\
			\end{example}

			\begin{example}
			\textbackslash begin \{ spacing \} \{ 2.0 \} \\
			.............. \\
			\textbackslash end \{ spacing \} \\
			\end{example}

		\end{frame}



	%	----------------------------------------------------------
	%		문자간의 간격 띄우기
	%	----------------------------------------------------------
		\begin{frame}[t,allowframebreaks]{문자간의 간격 띄우기}

			\begin{block} {문자간의 간격 띄우기}
			\begin{description}[1234567890]
			\item [$\sim$] 
			\item [\textbackslash hspace \{ 2cm \} ]
			\item [\textbackslash quad] 	(여백입력)\\
			\item [\textbackslash qquad] 	(여백입력 두배크기)\\
			\item [\textbackslash  ]textspace \\
			\item [\textbackslash ;] large space \\
			\item [\textbackslash \(>\)] medium space \\
			\item [\textbackslash ,] small space \\
			\item [\textbackslash !] negative space \\
			\end{description}

			\end{block}

		
		\end{frame}



	%	----------------------------------------------------------
	%		문단 정렬
	%	----------------------------------------------------------
		\begin{frame}[t]{문단 : 문단 정렬}

			\begin{block} {문단 정렬 :  범위 설정}
			\begin{description}[123456789012345]
			\item [오른쪽 정렬]	\textbackslash begin\{flushright\} text \textbackslash  end\{flushright\}
			\item [왼쪽 정렬]		\textbackslash begin\{flushleft\} text \textbackslash  end\{flushleft\}
			\item [가운데 정렬]	\textbackslash begin\{center\} text \textbackslash  end\{center\}
			\end{description}
			\end{block}

			\begin{block} {문단 정렬 : 선언 이후로 전부}
			\begin{description}[123456789012345]
			\item [오른쪽 정렬]	\textbackslash ragged right
			\item [왼쪽 정렬]		\textbackslash ragged left
			\item [가운데 정렬]	\textbackslash centering
			\end{description}
			\end{block}

		
		\end{frame}





	%	==========================================================
	%		개조식 문서
	%	----------------------------------------------------------


	%	----------------------------------------------------------
	%		List
	%	----------------------------------------------------------
		\begin{frame}[t]{List}

		
		\end{frame}


	%	----------------------------------------------------------
	%		itemize
	%	----------------------------------------------------------
		\begin{frame}[t]{itemize}

			\begin{block}{itemize}
			\end{block}

			\begin{block}{itemize 기호 모양 바꾸기}
			\end{block}



		\end{frame}

	%	----------------------------------------------------------
	%		enumerate
	%	----------------------------------------------------------
		\begin{frame}[t]{enumerate}

			\begin{block}{enumerate}
			\end{block}

			\begin{block}{enumerate 기호 모양 바꾸기}
			\end{block}


		\end{frame}


	%	----------------------------------------------------------
	%		description
	%	----------------------------------------------------------
		\begin{frame}[t]{description}

			\begin{block}{align : item의 정렬 방식}
			\begin{itemize}
			\item align=left
			\item align=right
			\end{itemize}
			\end{block}

			\begin{block}{style}
			\begin{itemize}
			\item style=standard
			\item style=unboxed
			\item style=nextline
			\item style=sameline,  leftmargin=2cm 충분히 크게 주어야 효과 있음
			\end{itemize}
			\end{block}


			\begin{block}{description 기호 모양 바꾸기}
			\end{block}


		\end{frame}






	%	----------------------------------------------------------
	%		TAB
	%	----------------------------------------------------------
		\begin{frame}[t]{TAB}

		\begin{block}{tabbing}
		\begin{itemize}
		\item[]	\textbackslash begin \{tabbing\}
		\item[]	\textbackslash hspace\{2cm\}	\textbackslash = 
				\textbackslash hspace\{2cm\}	\textbackslash = 
				\textbackslash hspace{2cm}	
				\textbackslash \textbackslash 

		\item[]	text	\textbackslash $>$
				text	\textbackslash $>$
				text	\textbackslash\textbackslash

		\item[]	text	\textbackslash $>$
				text	\textbackslash $>$
				text	\textbackslash\textbackslash

		\item[]	\textbackslash end \{tabbing\}
		\end{itemize}

		\end{block}


		\end{frame}


	%	----------------------------------------------------------
	%		TAB enum : 문제 풀이의 보기 항목
	%	----------------------------------------------------------
		\begin{frame}[t]{TAB enum}

		\begin{block}{tabenum}
		\begin{itemize}
		\item[]	\textbackslash begin\{tabenum\}[\textbackslash bfseries1]
		\item[]	\textbackslash tabenumitem 		\$z=\textbackslash displaystyle\textbackslash frac xy\$
		\item[]	\textbackslash tabenumitem 		\$z=\textbackslash displaystyle\textbackslash frac xy\$\textbackslash\textbackslash 
		\item[]	\textbackslash skipitem
		\item[]	\textbackslash tabenumitem 		\$z=\textbackslash displaystyle\textbackslash frac xy\$\textbackslash\textbackslash 
		\item[]	\textbackslash item 			\$z=\textbackslash displaystyle\textbackslash frac xy\$\textbackslash\textbackslash 
		\item[]	\textbackslash noitem 			\$z=\textbackslash displaystyle\textbackslash frac xy\$\textbackslash\textbackslash 
		\item[]	\textbackslash end\{tabenum\}
		\end{itemize}
		\end{block}

		\end{frame}












	%	----------------------------------------------------------	List : Vertical spacing

		\begin{frame}[c]{List : Vertical Horizontal spacing}

			\begin{block} {Vertical spacing}
			\begin{enumerate}
			\item	top sep		: 상단 간격
			\item	par top sep	: 상단 간격
			\item	par sep		: 아이템 내에서의 문단 간격
			\item	item sep		: 아이템간 간격
			\end{enumerate}
			\end{block}

			\begin{block} {Vertical spacing}
			\begin{enumerate}
			\item	상단		: top sep + par skip + par top sep
			\item	하단		: top sep + par skip + par top sep
			\item	아이템간	: item sep + par sep
 			\item	아이템 내 간격	: par sep
			\end{enumerate}
			\end{block}

	%	----------------------------------------------------------	List : Horizontal spacing

			\begin{block} {Horizontal spacing}
			\begin{enumerate}
			\item	left margin		: 왼쪽 여백
			\item	right margin		: 오른쪽 여백
			\item	list par indent	: 들여쓰기
			\item	label width		: 라벨의 폭
			\item	label sep			: 라벨의 본분과의 간격
			\item	item indent		: 아이템간 간격
			\end{enumerate}
			\end{block}

		
		\end{frame}



	%	----------------------------------------------------------
	%		Global settings
	%	----------------------------------------------------------
		\begin{frame}[c]{List ; Global settings}

			\begin{block} {Global settings}
			\begin{enumerate}
			\item	\textbackslash setlist [ enumerate, $<$ \textit{level} $>$ ] \{ $<$ \textit{format} $>$ \}
			\item	\textbackslash itemize [ itemize, $<$ \textit{level} $>$ ] \{ $<$ \textit{format} $>$ \}
			\item	\textbackslash description [ description, $<$ \textit{level} $>$ ] \{ $<$ \textit{format} $>$ \}
			\item	\textbackslash setlist [ $<$ \textit{level} $>$ ] \{ $<$ \textit{format} $>$ \}
			\end{enumerate}
			\end{block}

		
		\end{frame}



%	-------------------------------------------------------------------------------
%		Vertical and Horizontal spacing
%	-------------------------------------------------------------------------------
%		\setlist[enumerate,1]{labelindent=0.0em,leftmargin=8.0ex,rightmargin=2.0em,}
%		\setlist[enumerate,2]{labelindent=0.0em,leftmargin=4.0ex,rightmargin=2.0em,}
%		\setlist[enumerate,3]{labelindent=0.0em,leftmargin=3.0ex,rightmargin=2.0em,}

%	-------------------------------------------------------------------------------
%		Label
%	-------------------------------------------------------------------------------
%		\setlist[enumerate,1]{ label=\arabic*., ref=\arabic* }
%		\setlist[enumerate,1]{ label=\emph{\arabic*.}, ref=\emph{\arabic*} }
%		\setlist[enumerate,1]{ label=\textbf{\arabic*.}, ref=\textbf{\arabic*} }

%		\setlist[enumerate,2]{ label=\emph{\alph*}),ref=\theenumi.\emph{\alph*} }
%		\setlist[enumerate,3]{ label=\roman*), ref=\theenumii.\roman* }






















	%	==========================================================
	%		표 그리기
	%	----------------------------------------------------------


		\begin{frame}[plain]
		\centering
		\scalebox{10}{표}

		
		\end{frame}

	%	----------------------------------------------------------
	%		표
	%	----------------------------------------------------------
		\begin{frame}[t]{표}

		\begin{block}{표의 caption}
		\end{block}



		tabular height\\
		\textbackslash renewcommand \{ \textbackslash arraystretch\} \{1.2\} \\
		1 : default\\
		\textbackslash setlength \textbackslash minrowclearance \{2.4pt\} \\
		\textbackslash setlength \textbackslash extrarowheight \{5.0pt\} 

		\end{frame}


	%	----------------------------------------------------------
	%		tablex
	%	----------------------------------------------------------
		\begin{frame}[t]{표의 배치}

		\begin{block}{문서 내에서 표의 배치}
			\begin{itemize}
			\item[h] 이 자리, here
			\item[t] 문서의 최상단, top
			\item[b] 문서의 최하단, bottom
			\item[p] 별도위 표만 모아 놓은 장, page
			\end{itemize}
		\end{block}


		\begin{block}{표 내부에서 text의 배치}
			\begin{itemize}
			\item[c] 가운데 배치
			\item[l] 왼쪽 배치
			\item[r] 오른쪽 배치
			\item[p] 줄바꿈 가능하게, page graph\\
					중괄호를 붙여 크기를 강제로 지정할 수 있다.
			\end{itemize}
		\end{block}

		\begin{block}{표 내부에서 text의 배치}
			\begin{description}[12345678901234567890]
			\item[] \textbackslash usepackage\{array\}
			\item[]
			\item[p\{width\}] Top align, the same as usual.
			\item[m\{width\}] Middle align
			\item[b\{width\}] Bottom align
			\end{description}
		\end{block}

		clear page명령에 의해 표가 커거 강제로 뒤로 배치되는 것을 막을 수 있다.
		\end{frame}


	%	----------------------------------------------------------
	%		줄 내부 줄치기
	%	----------------------------------------------------------
		\begin{frame}[t]{표 내부 줄치기}

		\begin{block}{표 내부 줄치기}
		\begin{itemize}
		\item[] \textbackslash usepackage\{booktabs\}	\% toprule cmidrule midrule bottomrule  
		\item[]
		\item[]	\textbackslash top rule
		\item[]	\textbackslash mid rule
		\item[]	\textbackslash bottom rule
		\end{itemize}
		\end{block}


		\begin{columns}[t]
		\begin{column}{.4\textwidth}
				\begin{example}
					\begin{tabular}{llr}
					\toprule
					\multicolumn{2}{c}{Item} \\
					\cmidrule(r){1-2}
					Animal & Description & Price (\$)\\
					\midrule
					Gnat & per gram & 13.65 \\
					& each & 0.01 \\
					Gnu & stuffed & 92.50 \\
					Emu & stuffed & 33.33 \\
					Armadillo & frozen & 8.99 \\
					\bottomrule
					\end{tabular}
				\end{example}
		\end{column}

		\begin{column}{.5\textwidth}
				\begin{block}{code}
				\begin{itemize}
				\item[] \textbackslash begin \{ tabular \} \{ llr \}
				\item[] \textbackslash toprule
				\item[] \textbackslash multicolumn\{2\}\{c\}\{Item\} \textbackslash \textbackslash 
				\item[] \textbackslash cmidrule(r)\{1-2\}
				\item[] Animal \& Description \& Price (\$)\textbackslash \textbackslash 
				\item[] \textbackslash midrule
				\item[] Gnat \& per gram \& 13.65 \textbackslash \textbackslash 
				\item[] \& each \& 0.01 \textbackslash \textbackslash 
				\item[] Gnu \& stuffed \& 92.50 \textbackslash \textbackslash 
				\item[] Emu \& stuffed \& 33.33 \textbackslash \textbackslash 
				\item[] Armadillo \& frozen \& 8.99 \textbackslash \textbackslash 
				\item[] \textbackslash bottomrule
				\item[] \textbackslash end\{tabular\}
				\end{itemize}
				\end{block}
		\end{column}

		\end{columns}
		
		\end{frame}


	%	----------------------------------------------------------
	%		표 줄 간격 조정
	%	----------------------------------------------------------
		\begin{frame}[t]{표 줄 간격 조정}

		\begin{block}{표 줄간격 조정}
		\begin{description}[12345678901234567890]
		\item[asrray stretch] 		\textbackslash re new command \{ \textbackslash arraystretch \} \{ 1.2\}
		\item[extra row height]	\textbackslash usepackage \{ array \} \par
								\textbackslash set length \{ \textbackslash extrarowheight \}\{1.5pt\}
		\item[big strut]			\textbackslash usepackage \{ bigstrut\} \\
								\textbackslash bigstrut
		\end{description}
		\end{block}
		
		% ============================================
		\begin{columns}[t]
		% --------------------------------------------
		\begin{column}{0.3\textwidth}
		\begin{block}{asrray stretch}
		\begin{itemize}
		\item[] \textbackslash re new command \{ \textbackslash arraystretch \} \{ 1.2\}
		\end{itemize}
		\end{block}
		{
		\renewcommand{\arraystretch}{1.2}
		\begin{tabular}{|c|l|}
		\hline
		a & Row 1 \\ \hline
		b & Row 2 \\ \hline
		c & Row 2 \\
		d & Row 4 \\ \hline
		\end{tabular}
		}

		{
		\renewcommand{\arraystretch}{2.0}
		\begin{tabular}{|c|l|}
		\hline
		a & Row 1 \\ \hline
		b & Row 2 \\ \hline
		c & Row 2 \\
		d & Row 4 \\ \hline
		\end{tabular}
		}


		\end{column}

		% --------------------------------------------
		\begin{column}{0.3\textwidth}
		\begin{block}{Extra row height}
		\begin{itemize}
		\item[] \textbackslash usepackage \{ array \}
		\item[] \textbackslash set length \{ \textbackslash extrarowheight \}\{1.5pt\}
		\end{itemize}
		\end{block}

		{
		\setlength{\extrarowheight}{1.0em}
		\begin{tabular}{|l|l|}
		\hline
		a & Row 1 \\ \hline
		b & Row 2 \\ \hline
		c & Row 3 \\
		d & Row 4 \\ \hline
		\end{tabular}
		}

		{
		\setlength{\extrarowheight}{2.0em}
		\begin{tabular}{|l|l|}
		\hline
		a & Row 1 \\ \hline
		b & Row 2 \\ \hline
		c & Row 3 \\
		d & Row 4 \\ \hline
		\end{tabular}
		}

		\end{column}

		% --------------------------------------------
		\begin{column}{0.3\textwidth}
		\begin{block}{Big struts}
		\begin{itemize}
		\item[] \textbackslash usepackage \{ bigstrut\}
		\item[] \textbackslash bigstrut
		\end{itemize}
		\end{block}

		\begin{tabular}{|l|l|}
		\hline
		a & Row 1 \bigstrut \\ \hline
		b & Row 2 \bigstrut \\ \hline
		c & Row 2 \bigstrut[t] \\
		d & Row 4 \bigstrut[b] \\ \hline
		\end{tabular}


		\begin{tabular}{|l|l|}
		\hline
		a & Row 1 \bigstrut \\ \hline
		b & Row 2 \bigstrut \\ \hline
		c & Row 2 \bigstrut[t] \\ \hline
		d & Row 4 \bigstrut[b] \\ \hline
		\end{tabular}



		\end{column}
		% --------------------------------------------
		\end{columns}
		% ============================================


		\end{frame}

	%	----------------------------------------------------------
	%		표 열 간격 조정
	%	----------------------------------------------------------
		\begin{frame}[t]{표 열간격 조정}

		\begin{block}{표 열간격 조정}
		\begin{description}[12345678901234567890]
		\item[top col sep] 		\textbackslash set length \{\textbackslash tabcolsep\} \{ 10pt\}
		\end{description}
		\end{block}



		\end{frame}[t]{표 열간격 조정}

	%	----------------------------------------------------------
	%		열병합 행병합
	%	----------------------------------------------------------
		\begin{frame}[t]{열병합 행병합}

		\begin{block}{열병합 : 옆으로 병합}
		\begin{itemize}
		\item[]	\textbackslash multi column \{ number cols \} \{align \} \{ 내용 \}
		\item	align : l, c, r
		\end{itemize}
		\end{block}


		\begin{example}
		\begin{itemize}
		\item[]	\textbackslash multi column \{2\} \{c\} \{ 내용 \}
		\end{itemize}
		\end{example}

		\begin{block}{행병합 : 아래로 병합}
		\begin{itemize}
		\item[]	\textbackslash multi row \{number row \} \{width*\} \{ 내용 \}
		\item    width = *
		\end{itemize}
		\end{block}

		\begin{example}
		\begin{itemize}
		\item[]	\textbackslash multi row \{2\} \{*\} \{ 내용 \}
		\end{itemize}
		\end{example}

		
		\end{frame}




	%	----------------------------------------------------------
	%		tablex
	%	----------------------------------------------------------
		\begin{frame}[t]{table x}

		
			\begin{block}{code}
			\begin{itemize}
				\item[] \textbackslash usepackage\{booktabs\}	\% toprule cmidrule midrule bottomrule  
				\item[] \textbackslash usepackage\{tabularx\}
				\item[] 
				\item[] \textbackslash begin\{table\}[h]
				\item[] \textbackslash caption\{원형 구조물 허용공차(Allowable Tolerance)\}
				\item[] \textbackslash centering 
				\item[] \textbackslash begin\{tabularx\}\{0.8\textbackslash textwidth\}\{ X X c \}
				\item[] \textbackslash toprule
				\item[] 검사  항목\&허용  공차\&비고\textbackslash \textbackslash 
				\item[] \textbackslash midrule
				\item[] 평면  위치\&+ 15mm\textbackslash \textbackslash 
				\item[] 반          경\&+ 10mm  - 5mm\textbackslash \textbackslash 
				\item[] 바    닥    고\&+  5mm  - 2mm\textbackslash \textbackslash 
				\item[] 벽  체  두  께\&+ 10mm  - 5mm\textbackslash \textbackslash 
				\item[] 벽 체 천 단 고\&+  5mm  - 3mm\textbackslash \textbackslash 
				\item[] \textbackslash bottomrule
				\item[] \textbackslash end\{tabularx\} 
				\item[] \textbackslash label\{table-1\}
				\item[] \textbackslash end\{table\}
			\end{itemize}
			\end{block}{code}
	
			\begin{example}
				\begin{table}[h]
				\caption {원형 구조물 허용공차(Allowable Tolerance)}
				\centering 
				\begin{tabularx}{0.8\textwidth}{ X X c }
				\toprule
				검사  항목	&허용  공차&비고\\
				\midrule
				평면  위치		&+ 15mm\\
				반          경	&+ 10mm  - 5mm\\
				바    닥    고		&+  5mm  - 2mm\\
				벽  체  두  께		&+ 10mm  - 5mm\\
				벽 체 천 단 고		&+  5mm  - 3mm\\
				\bottomrule
				\end{tabularx} 
				\label{table-1}
				\end{table}
			\end{example}



		\end{frame}

	%	----------------------------------------------------------
	%		table y
	%	----------------------------------------------------------
		\begin{frame}[t]{table y}



		\end{frame}

	%	----------------------------------------------------------
	%		긴표
	%	----------------------------------------------------------
		\begin{frame}[t]{긴표}

		\begin{block}{긴표}
		\begin{itemize}
		\item[] \textbackslash usepackage\{longtable\}
		\item[] 
		\item[]	\textbackslash begin\{longtable\} \{ |c|c|c|c| \}
		\item[]	\hspace{2em} \textbackslash endfirsthead
		\item[]	\hspace{2em} \textbackslash endhead
		\item[]	\hspace{2em} \textbackslash endfoot
		\item[]	\hspace{2em} \textbackslash endlastfoot
		\item[]	
		\item[]	\textbackslash end\{longtable\} 
		\end{itemize}
		\end{block}



		
		\end{frame}


	%	----------------------------------------------------------
	%		표속에 각주 넣기
	%	----------------------------------------------------------
		\begin{frame}[t,shrink=0]{표속에 각주 넣기}

			\begin{block} {표속에 각주 넣기}
			\textbackslash footnotemark [ 번호 ] \\
			\textbackslash footnotetext [ 번호 ] \{ 각주 내용 \}
			\end{block}

			\begin{example}
			\begin{itemize}
				\item[]	\textbackslash begin\{table\}[!h]
				\item[]	\textbackslash caption\{페이지 바닥에 ˜각주를 표시하는 표\}
				\item[]	\textbackslash begin\{center\}
				\item[]	\textbackslash begin\{tabular\}\{|c|c|c|\}
				\item[]	\textbackslash hline
				\item[]	GDP \textbackslash footnotemark[1] \&
				\item[]	GDP \textbackslash footnotemark[2] \&
				\item[]	GDP \textbackslash footnotemark[3] \textbackslash \textbackslash
				\item[]	\textbackslash hline
				\item[]	\textbackslash end\{tabular\}
				\item[]	\textbackslash end\{center\}
				\item[]	\textbackslash label\{tab:pagefootnote\}
				\item[]	\textbackslash end\{table\}
				\item[]	
				\item[]	\textbackslash footnotetext[1]\{2007D 한국은행 ‰\}
				\item[]	\textbackslash footnotetext[2]\{2008D 한국은향 추정치 \}
				\item[]	\textbackslash footnotetext[3]\{2008D KDI추정치 \}
			\end{itemize}
			\end{example}



		\textbackslash end \{ table \} 이후에 \textbackslash footnotetext를 위치 시킨다. 

		\end{frame}




	%	==========================================================
	%		그림
	%	----------------------------------------------------------

		\begin{frame}[plain]
		\centering
		\scalebox{10}{그림}

		
		\end{frame}

	%	----------------------------------------------------------
	%		그림 문서에 넣기
	%	----------------------------------------------------------

		\begin{frame}[t]{그림 : 문서에 그림 넣기}

			\begin{block} {문서에 그림 넣기}
			\begin{itemize}
			\item[]	\textbackslash begin \{ figure \} [ where] 
			\item[]	\textbackslash centering
			\item[]	\textbackslash  caption \{ 그림 설명문 \}
			\item[]	\textbackslash  includegraphics \{ 그림파일명.확장자 \}
			\item[]	\textbackslash  label \{ fig:001 \}
			\item[]	\textbackslash end \{ figure \}
			\end{itemize}
			\end{block}

	%	----------------------------------------------------------	include graphics

			\begin{block} {그림 : 문서에 jpg 파일 넣기  include graphics}
			\begin{itemize}
			\item[]	\textbackslash  includegraphics \{ 그림파일명.확장자 \}
			\item[]	\textbackslash  includegraphics [ scale=0.9 ] \{ 그림파일명.확장자 \}
			\item[]	\textbackslash  includegraphics [ width=1.0\textbackslash textwidth ] \{ 그림파일명.확장자 \}
			\item[]	\textbackslash  includegraphics [ angle=value ] \{ 그림파일명.확장자 \}
			\end{itemize}
			\end{block}


	%	----------------------------------------------------------	include pdf

			\begin{block} {그림 : 문서에 pdf 파일 넣기 include pdf}
			\begin{itemize}
			\item[]	\textbackslash includepdf [ pages=-] \{그림파일명.pdf\}
			\item[]	\textbackslash includepdf [ pages=-, fitpaper=true] \{그림파일명.pdf\}
			\item[]	\textbackslash includepdf [ pages=-, scale=0.9] \{그림파일명.pdf\}
			\item[]	\textbackslash includepdf [ pages=-, frame=truee ] \{그림파일명.pdf\}
			\item[]	\textbackslash includepdf [ pages=-, landscape=false ] \{그림파일명.pdf\}
			\end{itemize}
			\end{block}

			page=- : 전체 페이지를 삽입\\
			fitpaper=trun : 전체 페이지에 배치

	%	---------------------------------------------------------- include path
			\begin{block} {include 파일경로}
			\begin{itemize}
			\item[]	\textbackslash includepdf \{./fig/그림파일명.pdf\}
			\item[]	\textbackslash graphicspath\{\{images/\}\}   

			\end{itemize}
			\end{block}

			그래픽 패스는 지정 가능하다.\\
			pdf 파일의 경로 지정은 ? (답) 그림과 똑 같다.

		\end{frame}























	%	==========================================================
	%		수식 
	%	----------------------------------------------------------

		\begin{frame}[plain]
		\centering
		\scalebox{10}{수식}

		
		\end{frame}




	%	----------------------------------------------------------
	%		수식 사용전 package 선언
	%	----------------------------------------------------------

		\begin{frame}[t]{수식}

			\begin{block} {수식 사용전 package 선언}
			\begin{itemize}
			\item[]	\textbackslash usepackage \{ amsmath \}
			\end{itemize}
			\end{block}

		
		\end{frame}



	%	----------------------------------------------------------
	%		수식 모드
	%	----------------------------------------------------------
		\begin{frame}[t, shrink=5]{수식 모드}

			\begin{block} {문단 내 배치}
				\begin{itemize}
				\item[]  \$ 수식 입력  \$
				\item[]  \textbackslash ( 수식 입력  \textbackslash )
				\item[]  \textbackslash begin \{ 수식 입력  \} \textbackslash end \{ math\}
				\end{itemize}
			\end{block}

			\begin{block} {별도 단락으로 배치 (별도의 한줄로 배치) }
				\begin{itemize}
				\item[]  \textbackslash [ 수식 입력 \textbackslash ]
				\item[]  \textbackslash begin \{ displaymath \} \textbackslash end \{ displaymath \}
				\item[]  \textbackslash begin \{ equation \} \textbackslash end \{ equation \} : 수식 번호를 가짐
				\end{itemize}
			\end{block}

			\begin{block} {하나의 수식이 너무 길어서 여러줄에 걸쳐 배치}
				\begin{itemize}
				\item[]  \textbackslash begin \{ multline \}  \textbackslash end \{ multline \}  : 
						\textbackslash \textbackslash로 줄바꿈
				\end{itemize}
			\end{block}

			\begin{block} {여러 수식들을 여러줄에 걸쳐 정렬 배치}
				\begin{itemize}
				\item[]  \textbackslash begin \{ align * \}  \textbackslash end \{ align * \}
						: \&로 정렬 기준, \textbackslash \textbackslash로 줄바꿈
						\textbackslash nonumber 로 수식번호 안 붙임
				\end{itemize}
			\end{block}

		
		\end{frame}


	%	----------------------------------------------------------
	%		수식의 정렬
	%	----------------------------------------------------------
		\begin{frame}[t]{수식의 정렬}

			\begin{block} {왼쪽 정렬}
			\end{block}

			\begin{block} {중간 정렬 : default}
			\end{block}

			\begin{block} {오른쪽 정렬}
			\end{block}

		
		\end{frame}





	%	----------------------------------------------------------
	%		수식 기본 연산 기호
	%	----------------------------------------------------------
		\begin{frame}[t]{수식 : 기본 연산 기호}

			\begin{block} {수식 : 기본 연산 기호}
			\begin{description}
			\item[더하기] 	$+$
			\item[빼기] 	$-$
			\item[곱하기] 	$\times$ ( \$ \textbackslash times \$ )
			\item[나누기] 	$\div$ (\$\textbackslash div\$)
			\item[] 		$\frac{1}{2}$ (\$\textbackslash frac\{1\}\{2\}\$)
			\item[] 		$\displaystyle\frac{1}{2}$
						(\$\textbackslash displaystyle\textbackslash frac\{1\}\{2\}\$)
			\end{description}
			\end{block}
	
	%	----------------------------------------------------------	수식 기본 기호
			\begin{block} {기본 기호}
			\begin{description}
			\item[$\therefore$] 	\$ \textbackslash therefore \$
			\end{description}
			\end{block}

		
		\end{frame}


	%	----------------------------------------------------------
	%		수식에서 한글 입력
	%	----------------------------------------------------------
		\begin{frame}[t]{수식에서 한글 입력}

			\begin{block} {수식에서 한글 입력}
			수식에서 한글 입력기 깨어지는 현상이 무조건 발생한다.\\
			이때는 한글 부분은 \textbackslash \{\}로 묶어서 처리하면 된다.
			\end{block}

		
		\end{frame}

	%	----------------------------------------------------------
	%		수식 미분
	%	----------------------------------------------------------
		\begin{frame}[t]{미분}

		
		\end{frame}


	%	----------------------------------------------------------
	%		수식 적분
	%	----------------------------------------------------------
		\begin{frame}[t]{적분}

		
		\end{frame}


	%	----------------------------------------------------------
	%		수식 행렬
	%	----------------------------------------------------------
		\begin{frame}[t]{행렬}

		
		\end{frame}







% 	================================================= chapter 	====================
%		box
%	-------------------------------------------------------------------------------

		\begin{frame}[plain]
		\centering
		\scalebox{10}{BOX}

		
		\end{frame}

	% ------------------------------------------------------------------------------
	%	par box
	% ------------------------------------------------------------------------------

		\begin{frame}[t]{parbox}

			\begin{block} {parbox}
			\textbackslash parbox[position][height][inner-pos]\{width\}\{text\}
			\end{block}

			\begin{block}{position}
			\begin{itemize}
			\item	t --- text is placed at the top of the box.
			\item	c --- text is centred in the box.
			\item	b --- text is placed at the bottom of the box.
			\item	s --- stretch vertically. The text must contain vertically stretchable space for this to work.
			\end{itemize}
			\end{block}

	% ------------------------------------------------------------------------------	mbox

			\begin{block} {mbox}
			\textbackslash mbox\{text\}
			\end{block}

			\begin{example}
			\mbox{mbox mbox mbox mbox mbox mbox mbox}
			\end{example}

	% ------------------------------------------------------------------------------	fbox

			\begin{block} {f box}
			\textbackslash fbox\{text\}
			\end{block}

			\begin{example}
			\fbox{mbox mbox mbox mbox mbox mbox mbox}
			\end{example}

	% ------------------------------------------------------------------------------	pbox

			\begin{block} {p box}
			\textbackslash pbox[b]\{\textbackslash textwidth\}\{my text\}
			\end{block}

			\begin{example}
			\end{example}

	% ------------------------------------------------------------------------------	save box

			\begin{block} {save box}
			\end{block}

	% ------------------------------------------------------------------------------	rotate box

			\begin{block} {rotate box}
			\end{block}

	% ------------------------------------------------------------------------------	colorbox and fcolorbox

			\begin{block} {colorbox}
			\end{block}


			\begin{block} {fcolorbox}
			\end{block}

		
		\end{frame}



	% ------------------------------------------------------------------------------	resize box

		\begin{frame}[t]{resize box}


			\begin{block} {resize box}
			\textbackslash resizebox\{수평 3em\}\{수직 2em\}\{문서 내용 Dunhill style\}

			\end{block}

			\begin{example}
		\resizebox{3em}{2em}{Dunhill style} \\
		\resizebox{4em}{2em}{Dunhill style} \\
		\resizebox{5em}{2em}{Dunhill style} \\
		\resizebox{30em}{2em}{글자 수평 수직 확대} \\
		\resizebox{10em}{3em}{글자 수평 수직 확대} \\

%
%		\resizebox{8em}{1ex}{Dunhill style} \\
%		\resizebox{8em}{2ex}{Dunhill style} \\
%		\resizebox{8em}{3ex}{Dunhill style} \\
%		\resizebox{8em}{4ex}{Dunhill style} \\
%		\resizebox{8em}{5ex}{Dunhill style} \\
%		\resizebox{8em}{6ex}{Dunhill style} \\
			\end{example}


	% ------------------------------------------------------------------------------ scale box

			\begin{block} {scale box}
			\textbackslash scalebox\{스케일 크기 1\}\{문서 내용 화이팅!\}
			\end{block}

			\begin{example}
		\scalebox{1}{화이팅!}\\
		\scalebox{2}{화이팅!}\\
		\scalebox{3}{화이팅!}\\
			\end{example}

	% ------------------------------------------------------------------------------ max size box

			\begin{block} {max size box}
			\begin{itemize}
			\item[] \textbackslash usepackage\{adjustbox\}
			\item[] \textbackslash framebox [width] [pos] \{text\}
			\end{itemize}
			\end{block}

	

	% ------------------------------------------------------------------------------	fancybox

			\begin{block} {scale box}
				\begin{itemize}
				\item double box
				\item oval box
				\item shadow box
				\end{itemize}
			\end{block}

	
		\end{frame}



	% ------------------------------------------------------------------------------
	%	makebox
	% ------------------------------------------------------------------------------
		\begin{frame}[t]{makebox}


			\begin{block} {make box}
			\textbackslash makebox [ width ] [ pos ] \{text\}
			\end{block}


			\begin{block} {position}
				\begin{itemize}
				\item	c : center
				\item	l : flushleft
				\item	r : flushright
				\item	s : spread
				\end{itemize}
			\end{block}

			\begin{example}
			c : \makebox[0.4\linewidth][c]	{makebox makebox }\\
			l : \makebox[0.4\linewidth][l]	{makebox makebox }\\
			r : \makebox[0.4\linewidth][r]	{makebox makebox }\\
			s : \makebox[0.4\linewidth][s]	{makebox makebox }\\
			\end{example}


		
		\end{frame}


	% ------------------------------------------------------------------------------
	%	framebox
	% ------------------------------------------------------------------------------
		\begin{frame}[t]{framebox}



			\begin{block} {frame box}
			\textbackslash framebox [width] [pos] \{text\}
			\end{block}

			\begin{block} {position}
				\begin{description}
				\item	[fboxsep]  the distance between the frame and the content.
				\item	[fboxrule] the thickness of the rule.
				\end{description}
			\end{block}



		
		\end{frame}


% 	================================================= chapter 	====================
%		mini page
%	-------------------------------------------------------------------------------
		\begin{frame}[t]{minipage}


			\begin{block} {minipage}
			\end{block}

		
		\end{frame}





	% ------------------------------------------------------------------------------
	%	boxed mini page
	% ------------------------------------------------------------------------------
		
		\begin{frame}[t]{boxedminipage}

			\begin{block} {boxedminipage}
			\end{block}

			\begin{example}
			\end{example}


		
		\end{frame}


































	%	==========================================================
	%		참고 문헌
	%	----------------------------------------------------------
	%		참고 문헌
	%	----------------------------------------------------------
		\begin{frame}[t]{참고문헌}

		
		\end{frame}


























































% ------------------------------------------------------------------------------
% End document
% ------------------------------------------------------------------------------
\end{document}


