%	-------------------------------------------------------------------------------
%
%
%
%
%
%
%
%
%	-------------------------------------------------------------------------------

%	\documentclass[10pt,xcolor=pdftex,dvipsnames,table]{beamer}
%	16:10
%	\documentclass[ aspectratio=1610, 10pt,blue,xcolor=pdftex,dvipsnames,table,handout]{beamer}
%	\documentclass[ aspectratio=1610, 10pt,blue,xcolor=pdftex,dvipsnames,table,handout,notes]{beamer}
%	16:9 
%	\documentclass[ aspectratio=169,  10pt,blue,xcolor=pdftex,dvipsnames,table,handout]{beamer}
%	\documentclass[ aspectratio=169,  10pt,blue,xcolor=pdftex,dvipsnames,table,handout,notes]{beamer}
%	14:9 
%	\documentclass[ aspectratio=149,  10pt,blue,xcolor=pdftex,dvipsnames,table,handout]{beamer}
	\documentclass[ aspectratio=149,  10pt,blue,xcolor=pdftex,dvipsnames,table,handout,notes]{beamer}

%	5:4
%	\documentclass[ aspectratio=54,   10pt,blue,xcolor=pdftex,dvipsnames,table,handout]{beamer}
%	4:3 default
%	\documentclass[ aspectratio=43, 10pt,blue,xcolor=pdftex,dvipsnames,table,handout]{beamer}
%	3:2 
% 	\documentclass[ aspectratio=32, 10pt,blue,xcolor=pdftex,dvipsnames,table,handout]{beamer}
		
		% Font Size
		%	default font size : 11 pt
		%	8,9,10,11,12,14,17,20
		%
		% 	put frame titles 
		% 		1) 	slideatop
		%		2) 	slide centered
		%
		%	navigation bar
		% 		1)	compress
		%		2)	uncompressed
		%
		%	Color
		%		1) blue
		%		2) red
		%		3) brown
		%		4) black and white	
		%
		%	Output
		%		1)  	[default]	
		%		2)	[handout]		for PDF handouts
		%		3) 	[trans]		for PDF transparency
		%		4)	[notes=hide/show/only]

		%	Text and Math Font
		% 		1)	[sans]
		% 		2)	[sefif]
		%		3) 	[mathsans]
		%		4)	[mathserif]


		%	---------------------------------------------------------	
		%	슬라이드 크기 설정 ( 128mm X 96mm )
		%	---------------------------------------------------------	
			\setbeamersize{text margin left=10mm}
			\setbeamersize{text margin right=10mm}

	%	========================================================== 	Package
		\usepackage{kotex}						% 한글 사용
		\usepackage{amssymb,amsfonts,amsmath}	% 수학 수식 사용
		\usepackage{color}						%
		\usepackage{colortbl}					%

		
		%	---------------------------------------------------------	
		%		유인물 출력 : 출력할대 조정해서 출력 할것
		%	---------------------------------------------------------	

			\usepackage{pgfpages}
%			\pgfpagesuselayout{2 on 1}[letterpaper]
%			\pgfpagesuselayout{4 on 1}[letterpaper]
%			\pgfpagesuselayout{8 on 1}[letterpaper]

%			\pgfpagesuselayout{resize to}[a4paper,landscape,border shrink=5mm]
%			\pgfpagesuselayout{resize to}[a4paper,border shrink=5mm]
%			\pgfpagesuselayout{2 on 1}[a4paper,border shrink=5mm]
%			----------------------------------------------------- 출력 시 설정	1
%			\pgfpagesuselayout{2 on 1}[a4paper]
%			\usecolortheme{seagull}	% 휜색
%			----------------------------------------------------- 출력 시 설정	2
			\pgfpagesuselayout{2 on 1}[a4paper,border shrink=5mm]
			\usecolortheme{dove}
%			----------------------------------------------------- 
%			\pgfpagesuselayout{4 on 1}[a4paper,border shrink=5mm]
%			\pgfpagesuselayout{8 on 1}[a4paper,border shrink=5mm]

			\usepackage{handoutWithNotes}
%			\pgfpagesuselayout{1 on 1 with notes}[a4paper,border shrink=5mm]
%			\pgfpagesuselayout{2 on 1 with notes}[a4paper,border shrink=5mm]
%			\pgfpagesuselayout{3 on 1 with notes}[a4paper,border shrink=5mm]
%			\pgfpagesuselayout{4 on 1 with notes}[a4paper,border shrink=5mm]

%			\pgfpagesuselayout{2 on 1}[letterpaper]
%			\pgfpagesuselayout{2 on 1}[letterpaper]
%			\pgfpagesuselayout{2 on 1}[letterpaper]



	%		========================================================= 	Theme

		%	---------------------------------------------------------	
		%	전체 테마
		%	---------------------------------------------------------	
		%	테마 명명의 관례 : 도시 이름
%			\usetheme{default}			%
%			\usetheme{Madrid}    		%
%			\usetheme{CambridgeUS}    	% -red, no navigation bar
			\usetheme{Antibes}			% -blueish, tree-like navigation bar

		%	----------------- table of contents in sidebar
%			\usetheme{Berkeley}		% -blueish, table of contents in sidebar
									% 개인적으로 마음에 듬

%			\usetheme{Marburg}			% - sidebar on the right
%			\usetheme{Hannover}		% 왼쪽에 마크
%			\usetheme{Berlin}			% - navigation bar in the headline
%			\usetheme{Szeged}			% - navigation bar in the headline, horizontal lines
%			\usetheme{Malmoe}			% - section/subsection in the headline

%			\usetheme{Singapore}
%			\usetheme{Amsterdam}

		%	---------------------------------------------------------	
		%	색 테마
		%	---------------------------------------------------------	
%			\usecolortheme{albatross}	% 바탕 파란
%			\usecolortheme{crane}		% 바탕 흰색
%			\usecolortheme{beetle}		% 바탕 회색
%			\usecolortheme{dove}		% 전체적으로 흰색
%			\usecolortheme{fly}		% 전체적으로 회색
			\usecolortheme{seagull}	% 휜색
%			\usecolortheme{wolverine}	& 제목이 노란색
%			\usecolortheme{beaver}

		%	---------------------------------------------------------	
		%	Inner Color Theme 			내부 색 테마 ( 블록의 색 )
		%	---------------------------------------------------------	

%			\usecolortheme{rose}		% 흰색
%			\usecolortheme{lily}		% 색 안 칠한다
%			\usecolortheme{orchid} 	% 진하게

		%	---------------------------------------------------------	
		%	Outter Color Theme 		외부 색 테마 ( 머리말, 고리말, 사이드바 )
		%	---------------------------------------------------------	

%			\usecolortheme{whale}		% 진하다
%			\usecolortheme{dolphin}	% 중간
%			\usecolortheme{seahorse}	% 연하다

		%	---------------------------------------------------------	
		%	Font Theme 				폰트 테마
		%	---------------------------------------------------------	
%			\usfonttheme{default}		
			\usefonttheme{serif}			
%			\usefonttheme{structurebold}			
%			\usefonttheme{structureitalicserif}			
%			\usefonttheme{structuresmallcapsserif}			



		%	---------------------------------------------------------	
		%	Inner Theme 				
		%	---------------------------------------------------------	

%			\useinnertheme{default}
			\useinnertheme{circles}		% 원문자			
%			\useinnertheme{rectangles}		% 사각문자			
%			\useinnertheme{rounded}			% 깨어짐
%			\useinnertheme{inmargin}			




		%	---------------------------------------------------------	
		%	이동 단추 삭제
		%	---------------------------------------------------------	
%			\setbeamertemplate{navigation symbols}{}

		%	---------------------------------------------------------	
		%	문서 정보 표시 꼬리말 적용
		%	---------------------------------------------------------	
%			\useoutertheme{infolines}


			
	%	---------------------------------------------------------- 	배경이미지 지정
%			\pgfdeclareimage[width=\paperwidth,height=\paperheight]{bgimage}{./fig/Chrysanthemum.jpg}
%			\setbeamertemplate{background canvas}{\pgfuseimage{bgimage}}

		%	---------------------------------------------------------	
		% 	본문 글꼴색 지정
		%	---------------------------------------------------------	
%			\setbeamercolor{normal text}{fg=purple}
%			\setbeamercolor{normal text}{fg=red!80}	% 숫자는 투명도 표시


		%	---------------------------------------------------------	
		%	itemize 모양 설정
		%	---------------------------------------------------------	
%			\setbeamertemplate{items}[ball]
%			\setbeamertemplate{items}[circle]
%			\setbeamertemplate{items}[rectangle]


		%	---------------------------------------------------------	
		%	상자 모양새 설정
		%	---------------------------------------------------------	
%			\setbeamertemplate{blocks}[rounded,shadow=true]
%			\begin{block}
%			\begin{theorem}
%			\begin{lemma}
%			\begin{proof}
%			\begin{corollary}
%			\begin{example}
%			\begin{exampleblock}
%			\begin{alertblock}




		\setbeamercovered{dynamic}






% ------------------------------------------------------------------------------
% Begin document (Content goes below)
% ------------------------------------------------------------------------------
	\begin{document}
	

			\title{LATEX}
			\subtitle{사용설명서}
			\author{김대희}
			\date[2015.06.30]{2015년 6월}
			\institute[KTS]{(주)서영엔지니어링 \url{http://symsone.seoyeong.co.kr/}}



	%	==========================================================
	%
	%	----------------------------------------------------------
		\begin{frame}[plain]
		\titlepage
		\end{frame}

		\begin{frame}[plain]
		\end{frame}


	%	==========================================================
	%		문서 클래스
	%	----------------------------------------------------------


	%	----------------------------------------------------------
	%		문서 클래스
	%	----------------------------------------------------------
		\begin{frame}[t,shrink=00]{문서 클래스}


			\begin{block} {장, 절의 설정}
			\begin{description}[1234567890]
			\item [article] 과학 학술지, 프리젠테이션, 짧은 보고서, 프로그램 문서, 초대장 등에 쓰이는 아티클용 클래스
			\item [proc article] 클래스에 기초한 프로시딩을 위한 클래스 
			\item [minimal] 최소 문서 양식 클래스. 페이지 크기와 기본 글꼴만을 설정한다. 주로 디버깅을 위하여 사용함.
			\item [report] 여러 장(chapter)으로 이루어진 긴 보고서, 작은 책, 박사학위 논문 등에 쓰이는 클래스.
			\item [book] 진짜 책을 만들기 위한 클래스.
			\item [slides] 슬라이드 제작용 클래스. 
			\end{description}
			\end{block}



		\note[item]{}
		\end{frame}


	%	----------------------------------------------------------
	%		문서 클래스 옵션
	%	----------------------------------------------------------
		\begin{frame}[t,shrink=0]{문서 클래스}


			\begin{block} {문서 클래스 옵션}
			\begin{description}[1234567890]
			\item [10pt] 11pt, 12pt 문서 기본 글꼴 크기를 설정한다.
			\item [letterpaper] a4paper,a5paper, b5paper, executivepaper,legalpaper
			\item [fleqn] 수식을 가운데 정렬이 아닌 왼쪽 정렬로 식자한다.
			\item [leqno] 수식 번호를 수식의 오른쪽이 아닌 왼쪽에 표시되도록 한다.
			\item [titlepage] notitlepage 표지 뒤에 새로운 페이지를 시작하도록 할 것인지 지정한다. report와 book은 새 페이지를 만든다.
			\item [onecolumn] twocolumn 문서를 1단 또는 2단으로 조판하도록 지시한다.
			\item [twoside] oneside 양면인쇄용 출력물 생성.\\ 단면(article,report) 양면 (book)
			\item [landscape] 레이아웃을 가로가 긴 형식(landscape)으로 변경한다.
			\item [openright] openany 새로운 장을 홀수쪽에서 시작.\\
								book 클래스에서는 홀수쪽에서 시작하는 것이 기본값이다.
			\end{description}
			\end{block}



		\note[item]{}
		\end{frame}



	%	==========================================================
	%		패키지
	%	----------------------------------------------------------
		\begin{frame}[t]{패키지}

		\note{패키지}
		\end{frame}



	%	==========================================================
	%		쪽 양식
	%	----------------------------------------------------------
		\begin{frame}[t]{쪽양식}

			\begin{block} {쪽양식}
			\begin{description}[1234567890]
			\item [\textbf{plain}] 쪽 번호를 쪽의 아래쪽 바닥글에 중앙정렬하여 찍는다.\\
								쪽 양식의 기본값이다.
			\item [\textbf{headings}] 현재 장 표제와 쪽 번호를 각 쪽의 머리글에 적는다.\\
								바닥글은 비운다.
			\item [\textbf{empty}] 머리글과 바닥글을 모두 비운다.
			\end{description}
			\end{block}

		\note[item]{}
		\end{frame}


	%	==========================================================
	%		장, 절의 설정
	%	----------------------------------------------------------
		\begin{frame}[t]{장, 절의 설정}

			\begin{block} {장, 절의 설정}
			\begin{enumerate}
			\item	\textbackslash part
			\item	\textbackslash chapter
			\item	\textbackslash section
			\item	\textbackslash sub section
			\item	\textbackslash sub sub section
			\item	\textbackslash paragraph
			\item	\textbackslash sub paragraph
			\end{enumerate}
			\end{block}

		\note[item]{}
		\end{frame}




	%	==========================================================
	%		title page
	%	----------------------------------------------------------
		\begin{frame}[t]{표지 작성}


			\begin{block} {표지작성}
			\begin{description}[1234567890]
			\item [\textbackslash title] 	문서 제목
			\item [\textbackslash author]	문서 저자
			\item [\textbackslash date] 	작성일
			\item [\textbackslash maketitle] 타이틀 표시
			\end{description}
			\end{block}

		\note[item]{표지의 표시 순서 변경 방법}
		\end{frame}




	%	==========================================================
	%		초록 작성
	%	----------------------------------------------------------
		\begin{frame}[t]{초록 작성}

			\begin{block} {초록 작성}
			\begin{description}[12345678901234567]
			\item [\textbackslash begin\{abstract\}] 	문서 제목
			\item [초록 내용]						초록 내용
			\item [\textbackslash end\{abstract\}] 	작성일
			\end{description}
			\end{block}


		\note[item]{}
		\end{frame}




	%	==========================================================
	%		목차 작성
	%	----------------------------------------------------------
		\begin{frame}[t]{목차 작성}

			\begin{block} {목차 작성}
			\begin{description}[12345678901234567]
			\item [\textbf{\textbackslash table of contents}] 	문서 내용 목차
			\item [\textbf{\textbackslash list of figures}]		그림 목차
			\item [\textbf{\textbackslash list of tables}] 		표 목차
			\end{description}
			\end{block}

			\begin{block} {목차에서 페이지 나누기}
			\textbackslash addtocontents \{ toc \} \{ \textbackslash protect \textbackslash newpage \}
			\end{block}

		\note[item]{}
		\end{frame}

	%	----------------------------------------------------------
	%		목차 작성 : 목차 counter
	%	----------------------------------------------------------
		\begin{frame}[t]{목차 작성 : counter}

			\begin{block} {\textbackslash setcounter \{ tocdepth \} }
			\end{block}

			\begin{block} {\textbackslash setcounter \{ secnumdepth \} \{ n \} }
			\end{block}


			\begin{center}
			\begin{table}
			\begin{tabular}{ l l }
				\hline
				부(part)				&-1\\
				장(chapter)			&0\\
				절(section)			&1\\
				소절(subsection)		&2\\
				소소절(subsubsection)	&3\\
				문단(paragraph)		&4\\
				소문단(subparagraph)	&5\\
				\hline
			\end{tabular}
			\end{table}
			\end{center}


		\note[item]{}
		\end{frame}


	%	==========================================================
	%		주석문 처리
	%	----------------------------------------------------------

		\begin{frame}[t]{주석문 처리}

			\begin{block} {주석문 처리}
			\textbackslash \%
			\end{block}

		\note[item]{}
		\end{frame}


	%	----------------------------------------------------------
	%		각주
	%	----------------------------------------------------------

		\begin{frame}[t]{각주}

			\begin{block} {각주}
			\textbackslash footnote \{ 각주 내용 \}
			\end{block}


		\note[item]{}
		\end{frame}



	%	----------------------------------------------------------
 	%		난외주
	%	----------------------------------------------------------

		\begin{frame}[t]{난외주}

			\begin{block} {난외주}
			\textbackslash marginpar \{ 난외주 내용 \}
			\end{block}

		\note[item]{}
		\end{frame}


	%	----------------------------------------------------------
	%		인용문
	%	----------------------------------------------------------

		\begin{frame}[t]{인용문}

			\begin{block} {인용문}
			\begin{itemize}
			\item[]	\textbackslash begin \{ quote \} ~인용문 내용 \textbackslash end \{ quote \} \\
			\item[]	\textbackslash begin \{ quotation \} ~인용문 내용 \textbackslash end \{ quotation \} 
			\end{itemize}
			\end{block}

			\begin{example}
				\begin{quote} 인용문 내용 \end{quote} 
				\begin{quotation} 인용문 내용 \end{quotation}
			\end{example}


		\note[item]{}
		\end{frame}





	%	----------------------------------------------------------
	%		 상호 참조
	%	----------------------------------------------------------

		\begin{frame}[t]{상호참조}

			\begin{block} {상호참조}
				\begin{itemize}
				\item \textbackslash label \{참조 기호 \}
				\item \textbackslash ref \{참조할 기호 \}
				\item \textbackslash pageref \{참조할 기호 \}
				\end{itemize}
			\end{block}

		\note[item]{}
		\end{frame}


	%	----------------------------------------------------------
	%		하이퍼링크
	%	----------------------------------------------------------

		\begin{frame}[t]{하이퍼링크}

			\begin{block} {하이퍼링크}
				\begin{itemize}
				\item [] \textbackslash  url\{하이퍼링크\}																\end{itemize}
			\end{block}

			\begin{example}
				\begin{itemize}
				\item \textbackslash url\{http://www.band.us/\#/band/53125310\} 
				\item \textbackslash url\{http://www.seoyeong.co.kr/\}
				\item \textbackslash url\{http://symsone.seoyeong.co.kr\}
				\item \textbackslash url\{http://syerp.seoyeong.co.kr/\}						
				\end{itemize}
			\end{example}

				\url{http://www.seoyeong.co.kr/}

		\note[item]{}
		\end{frame}





	%	==========================================================
	%		Page Size
	%	----------------------------------------------------------
		\begin{frame}[c]{Page size}


			\begin{block} {Page size}
			\begin{description}[12345678901234567890]
			\item	[a4paper] 210mm $\times$ 297mm
			\item	[a5paper]
			\item	[b5paper]
			\item	[letterpaper]
			\item	[legalpaper]
			\item	[executivepaper]
			\end{description}
			\end{block}


			\begin{example}
			\begin{itemize}
			\item[]	\textbackslash documentclass[a4paper,landscape,12pt]\{article\}
			\item[]	\textbackslash documentclass[a5paper,landscape,12pt]\{article\}
			\end{itemize}
			\end{example}


		\note[item]{}
		\end{frame}

	%	----------------------------------------------------------
	%		Page Size : geometry package ( 전체 여백을 지정한다. )
	%	----------------------------------------------------------
		\begin{frame}[c,allowframebreaks]
		\frametitle{geometry package}

			\begin{block} {geometry package}
			\textbackslash usepackage
						[ top=0.0mm, \\
						\hspace{6em} left=0.0mm, \\
						\hspace{6em} bottom=0.0mm, \\
						\hspace{6em} righr=0.0mm] \\
						\hspace{6em} \{geometry\}
			\end{block}
		\end{frame}

	%	----------------------------------------------------------
	%		Page Size : geometry package ( 전체 여백을 지정한다. )
	%	----------------------------------------------------------
		\begin{frame}[c,shrink=10]
		\frametitle{geometry package}

			\begin{block} {geometry 옵션 인자}
			\begin{description}[12345678901234567890]
			\item	[paperwidth]		=25cm
			\item	[paperheight]		=35cm
			\item	[papersize]		=\{25cm,35cm\}
			\item	[width]			=20cm \% total width
			\item	[heigth]			=30cm \% total heigth
			\item	[total]			=\{20cm,30cm\}
			\item	[textwidth]		=18cm \% width - marginpar
			\item	[textheight]		=25cm \% heigth - header - footer
			\item	[body]			=\{18cm,25cm\}
			\item	[left]			=3cm \% left margin
			\item	[right]			=1.5cm \% right margin
			\item	[hmargin]			=\{3cm,2cm\}
			\item	[top]			=2cm \% top margin
			\item	[bottom]			=3cm \% bottom margin
			\item	[vmargin]			=\{2cm,3cm\}
			\item	[marginparwidth]	=2cm
			\item	[head]			=1cm \% header space
			\end{description}
			\end{block}




		\end{frame}





	%	==========================================================
	%		문단 모양
	%	----------------------------------------------------------


	%	----------------------------------------------------------
	%		Line and page Breaking
	%	----------------------------------------------------------
		\begin{frame}[t]{페이지 나누기 줄바꾸기 Line and page Breaking}

			\begin{block} {Line and page Breaking}
			\begin{enumerate}
			\item	\textbackslash clear page
			\item	\textbackslash clear double page
			\item	\textbackslash hyphenation
			\item	\textbackslash line break
			\item	\textbackslash new line
			\item	\textbackslash no line break
			\item	\textbackslash no page break
			\item	\textbackslash page break
			\end{enumerate}
			\end{block}

		\note[item]{}
		\end{frame}

	%	----------------------------------------------------------
	%		문단 첫줄 들여쓰기
	%	----------------------------------------------------------
		\begin{frame}[t]{문단 첫줄 들여쓰기}

			\begin{block} {문단 첫줄 들여쓰기}
			\textbackslash par indent
			\end{block}

			\begin{example}
			\textbackslash setlength \{ \textbackslash parindent \} \{ 0.0cm \}\\
			\end{example}

		\note[item]{}
		\end{frame}

	%	----------------------------------------------------------
	%		문단과 문단 사이의 간격
	%	----------------------------------------------------------
		\begin{frame}[t]{문단과 문단 사이의 간격}

			\begin{block} {문단과 문단 사이의 간격}
			\textbackslash par skip
			\end{block}

			\begin{example}
			\textbackslash setlength \{ \textbackslash parskip \} \{ 0.0pt \}\\
			\textbackslash setlength \{\textbackslash parskip \} \{ 1cm plus 4mm minus\}	\\		
			\end{example}

			\begin{example}
			\textbackslash small skip
			\textbackslash med skip
			\textbackslash big skip
			\end{example}

		\note[item]{}
		\end{frame}


	%	----------------------------------------------------------
	%		문단내에서의 줄간격
	%	----------------------------------------------------------
		\begin{frame}[t,allowframebreaks]{문단내에서의 줄간격}

			\begin{block} {문단내에서의 줄간격}
			\textbackslash usepackage \{ setspace \} \\
			\begin{itemize}
			\item[]	\textbackslash singlespacing 
			\item[]	\textbackslash onehalfspacing 
			\item[]	\textbackslash doublespacing 
			\item[]	\textbackslash setstretch \{ $<$ $>$ \} 
			\item[]	\textbackslash linespread \{ $<$ factor $>$ \}
			\end{itemize}

			\end{block}

			\begin{example}
			\textbackslash linespread \{ 1.6 \} : double-spacing \\
			\textbackslash linespread \{ 1.3 \} : one-and-a half spacing 
			\end{example}
	
			\newpage
			\begin{example}
			\textbackslash begin \{ doublespace \} \\
			.............. \\
			\textbackslash end \{ doublespace \} \\
			\end{example}

			\begin{example}
			\textbackslash begin \{ spacing \} \{ 2.0 \} \\
			.............. \\
			\textbackslash end \{ spacing \} \\
			\end{example}

		\end{frame}



	%	----------------------------------------------------------
	%		문자간의 간격 띄우기
	%	----------------------------------------------------------
		\begin{frame}[t,allowframebreaks]{문자간의 간격 띄우기}

			\begin{block} {문자간의 간격 띄우기}
			\begin{enumerate}

			\item $\sim$\\
			\item \textbackslash hspace \{ 2cm \} \\
			\item \textbackslash quad 	(여백입력)\\
			\item \textbackslash qquad 	(여백입력 두배크기)\\
			\item \textbackslash textspace \\
			\item \textbackslash large space \\
			\item \textbackslash medium space \\
			\item \textbackslash small space \\
			\item \textbackslash negative space \\
			\end{enumerate}

			\end{block}

		\note[item]{}
		\end{frame}


	%	----------------------------------------------------------
	%		미리 정의된 문자열
	%	----------------------------------------------------------
		\begin{frame}[t,allowframebreaks]{미리 정의된 문자열}

			\begin{center}
			\rowcolors{1}{blue!10}{yellow!10}
			\begin{table}
			\begin{tabular}{ l l l  }
				명령어	&사용예	&설명\\
				\hline
				\textbackslash today 	&\today 		&현재 사용 언어에서의 현재 날짜 표기\\
				\textbackslash TeX 	&\TeX 		&최고의 조판 시스템의 이름\\
				\textbackslash LaTeX 	&\LaTeX 		&지금 우리가 배우고 있는 것의 이름\\
				\textbackslash LaTeXe 	&\LaTeXe		&LATEX의 최신판\\
			\end{tabular}
			\end{table}
			\end{center}

		\note[item]{}
		\end{frame}


	%	----------------------------------------------------------
	%		특수문자
	%	----------------------------------------------------------
		\begin{frame}[t,allowframebreaks]{특수문자}

			\begin{columns}[t]
			\begin{column}{.4\textwidth}
			\begin{block} {특수문자}
			\begin{enumerate}
			\item \# : \textbackslash \# \\
			\item \$ : \textbackslash \$ \\
			\item \% : \textbackslash \% \\
			\item \& : \textbackslash \& \\
			\item \_ : \textbackslash \_ \\
			\item \{ : \textbackslash \{ \\
			\item \} : \textbackslash \} \\
			\item \textbackslash  : \textbackslash textbackslash
			\end{enumerate}
			\end{block}
			\end{column}

			\begin{column}{.4\textwidth}
			\begin{block} {특수문자}
			\begin{enumerate}
			\item \_ : \textbackslash \_ \\
			\item \^{} : \textbackslash \^{} \\
			\item $\sim$ : \$ \textbackslash sim \$
			\end{enumerate}
			\end{block}
			\end{column}

			\end{columns}
		\end{frame}

	%	----------------------------------------------------------
		\begin{frame}[t,allowframebreaks]{특수문자 : 수학기호로 처리 }

			\begin{columns}[t]
			\begin{column}{.4\textwidth}
			\begin{block} {특수문자}
			\begin{enumerate}
			\item $\cdot$ 		: \$ $\cdot$ \$ \\
			\item $\circ$ 		: \$ $\circ$ \$ \\
			\item $\bullet$ 		: \$ $\bullet$ \$ \\
			\item $\centerdot$		: \$ $\centerdot$ \$ \\
			\item $\div$ 			: \$ $\div$ \$ \\
			\item $\times$ 		: \$ $\times$ \$ \\
			\end{enumerate}
			\end{block}
			\end{column}

			\begin{column}{.4\textwidth}
			\begin{block} {특수문자}
%			\begin{enumerate}
%			\item \approx		:  	\\
%			\item \neq		: 	\\
%			\item \infty		: 	\\
%			\end{enumerate}
			\end{block}
			\end{column}
			\end{columns}


			\textcircled{\small}\\

		\end{frame}





	%	----------------------------------------------------------
	%		선 그리기
	%	----------------------------------------------------------
		\begin{frame}[t]{선 그리기}

			\begin{block} {선 그리기}
			\textbackslash rule \{ \textbackslash linewidth \} \{ 두께 \}
			\end{block}

			\begin{example}
			\textbackslash rule \{ 4cm \} \{ 2mm \}
			\end{example}

			\rule{4cm}{2mm}

		\note[item]{}
		\end{frame}




	%	==========================================================
	%		글꼴 모양
	%	----------------------------------------------------------



	%	----------------------------------------------------------
	%		글꼴 모양
	%	----------------------------------------------------------


		\begin{frame}[t,allowframebreaks]{글꼴 모양}

			\begin{table}
			\begin{tabular}{ l l l  }
				명령어	&환경	&결과\\
				\hline
				\textbackslash textnarmal 	&textnarmal 	&\textnormal{결과}\\
				\textbackslash textit 		&itshape		&\textit{결과}\\
				\textbackslash emph 		&없음		&\emph{결과}\\
				\textbackslash textbf 		&bfseries		&\textbf{결과}\\
				\textbackslash underline 	&밑줄		&\underline{결과}\\
			\end{tabular}
			\end{table}

		\note[item]{}
		\end{frame}




	%	==========================================================
	%		글꼴 크기
	%	----------------------------------------------------------



		\begin{frame}[t,allowframebreaks]{글꼴 크기}

			\begin{table}
			\begin{tabular}{ l l l  }
				명령어		&결과\\
				\hline
				\textbackslash tiny			&\tiny{결과}\\
				\textbackslash scriptsize		&\scriptsize{결과}\\
				\textbackslash footnotesize		&\footnotesize{결과}\\
				\textbackslash normalsize		&\normalsize{결과}\\
				\textbackslash large			&\large{결과}\\
				\textbackslash Large			&\Large{결과}\\
				\textbackslash LARGE			&\LARGE{결과}\\
				\textbackslash huge			&\huge{결과}\\
				\textbackslash Huge			&\Huge{결과}\\
			\end{tabular}
			\end{table}

		\note[item]{}
		\end{frame}









	%	==========================================================
	%		개조식 문서
	%	----------------------------------------------------------


	%	----------------------------------------------------------
	%		List
	%	----------------------------------------------------------
		\begin{frame}[t]{List}

		\note[item]{}
		\end{frame}


	%	----------------------------------------------------------
	%		itemize
	%	----------------------------------------------------------
		\begin{frame}[t]{itemize}

		\end{frame}


	%	----------------------------------------------------------
	%		itemize 기호 모양 바꾸기
	%	----------------------------------------------------------
		\begin{frame}[t]{itemize 기호 모양 바꾸기}

		\end{frame}


	%	----------------------------------------------------------
	%		enumerate
	%	----------------------------------------------------------
		\begin{frame}[t]{enumerate}

		\end{frame}

	%	----------------------------------------------------------
	%		enumerate 기호 모양 바꾸기
	%	----------------------------------------------------------
		\begin{frame}[t]{enumerate 기호 모양 바꾸기}

		\end{frame}


	%	----------------------------------------------------------
	%		description
	%	----------------------------------------------------------
		\begin{frame}[t]{description}

			\begin{block}{align : item의 정렬 방식}
			\begin{itemize}
			\item align=left
			\item align=right
			\end{itemize}
			\end{block}

			\begin{block}{style}
			\begin{itemize}
			\item style=standard
			\item style=unboxed
			\item style=nextline
			\item style=sameline  leftmargin=2cm 충분히 크게 주어야 효과 있음
			\end{itemize}
			\end{block}



		\end{frame}


	%	----------------------------------------------------------
	%		description
	%	----------------------------------------------------------
		\begin{frame}[t]{description 기호 모양 바꾸기}

		\end{frame}




	%	----------------------------------------------------------
	%		TAB
	%	----------------------------------------------------------
		\begin{frame}[t]{TAB}

		\begin{block}{tabbing}
		\begin{itemize}
		\item[]	\textbackslash begin \{tabbing\}
		\item[]	\textbackslash hspace\{2cm\}	\textbackslash = 
				\textbackslash hspace\{2cm\}	\textbackslash = 
				\textbackslash hspace{2cm}	
				\textbackslash \textbackslash 

		\item[]	text	\textbackslash $>$
				text	\textbackslash $>$
				text	\textbackslash\textbackslash

		\item[]	text	\textbackslash $>$
				text	\textbackslash $>$
				text	\textbackslash\textbackslash

		\item[]	\textbackslash end \{tabbing\}
		\end{itemize}

		\end{block}


		\end{frame}


	%	----------------------------------------------------------
	%		TAB enum : 문제 풀이의 보기 항목
	%	----------------------------------------------------------
		\begin{frame}[t]{TAB enum}

		\begin{block}{tabenum}
		\begin{itemize}
		\item[]	\textbackslash begin\{tabenum\}[\textbackslash bfseries1]
		\item[]	\textbackslash tabenumitem 		\$z=\textbackslash displaystyle\textbackslash frac xy\$
		\item[]	\textbackslash tabenumitem 		\$z=\textbackslash displaystyle\textbackslash frac xy\$\textbackslash\textbackslash 
		\item[]	\textbackslash skipitem
		\item[]	\textbackslash tabenumitem 		\$z=\textbackslash displaystyle\textbackslash frac xy\$\textbackslash\textbackslash 
		\item[]	\textbackslash item 			\$z=\textbackslash displaystyle\textbackslash frac xy\$\textbackslash\textbackslash 
		\item[]	\textbackslash noitem 			\$z=\textbackslash displaystyle\textbackslash frac xy\$\textbackslash\textbackslash 
		\item[]	\textbackslash end\{tabenum\}
		\end{itemize}
		\end{block}

		\end{frame}












	%	----------------------------------------------------------
	%		List : Vertical spacing
	%	----------------------------------------------------------
		\begin{frame}[c]{List : Vertical spacing}

			\begin{block} {Vertical spacing}
			\begin{enumerate}
			\item	top sep		: 상단 간격
			\item	par top sep	: 상단 간격
			\item	par sep		: 아이템 내에서의 문단 간격
			\item	item sep		: 아이템간 간격
			\end{enumerate}
			\end{block}

			\begin{block} {Vertical spacing}
			\begin{enumerate}
			\item	상단		: top sep + par skip + par top sep
			\item	하단		: top sep + par skip + par top sep
			\item	아이템간	: item sep + par sep
 			\item	아이템 내 간격	: par sep
			\end{enumerate}
			\end{block}

		\note[item]{}
		\end{frame}



	%	----------------------------------------------------------
	%		List : Horizontal spacing
	%	----------------------------------------------------------
		\begin{frame}[c]{List : Horizontal spacing}

			\begin{block} {Horizontal spacing}
			\begin{enumerate}
			\item	left margin		: 왼쪽 여백
			\item	right margin		: 오른쪽 여백
			\item	list par indent	: 들여쓰기
			\item	label width		: 라벨의 폭
			\item	label sep			: 라벨의 본분과의 간격
			\item	item indent		: 아이템간 간격
			\end{enumerate}
			\end{block}

		\note[item]{}
		\end{frame}



	%	----------------------------------------------------------
	%		Global settings
	%	----------------------------------------------------------
		\begin{frame}[c]{List ; Global settings}

			\begin{block} {Global settings}
			\begin{enumerate}
			\item	\textbackslash setlist [ enumerate, $<$ \textit{level} $>$ ] \{ $<$ \textit{format} $>$ \}
			\item	\textbackslash itemize [ itemize, $<$ \textit{level} $>$ ] \{ $<$ \textit{format} $>$ \}
			\item	\textbackslash description [ description, $<$ \textit{level} $>$ ] \{ $<$ \textit{format} $>$ \}
			\item	\textbackslash setlist [ $<$ \textit{level} $>$ ] \{ $<$ \textit{format} $>$ \}
			\end{enumerate}
			\end{block}

		\note[item]{}
		\end{frame}



%	-------------------------------------------------------------------------------
%		Vertical and Horizontal spacing
%	-------------------------------------------------------------------------------
%		\setlist[enumerate,1]{labelindent=0.0em,leftmargin=8.0ex,rightmargin=2.0em,}
%		\setlist[enumerate,2]{labelindent=0.0em,leftmargin=4.0ex,rightmargin=2.0em,}
%		\setlist[enumerate,3]{labelindent=0.0em,leftmargin=3.0ex,rightmargin=2.0em,}

%	-------------------------------------------------------------------------------
%		Label
%	-------------------------------------------------------------------------------
%		\setlist[enumerate,1]{ label=\arabic*., ref=\arabic* }
%		\setlist[enumerate,1]{ label=\emph{\arabic*.}, ref=\emph{\arabic*} }
%		\setlist[enumerate,1]{ label=\textbf{\arabic*.}, ref=\textbf{\arabic*} }

%		\setlist[enumerate,2]{ label=\emph{\alph*}),ref=\theenumi.\emph{\alph*} }
%		\setlist[enumerate,3]{ label=\roman*), ref=\theenumii.\roman* }


	%	==========================================================
	%		표 그리기
	%	----------------------------------------------------------


		\begin{frame}[plain]
		\centering
		\scalebox{10}{표}

		\note[item]{}
		\end{frame}

	%	----------------------------------------------------------
	%		표
	%	----------------------------------------------------------
		\begin{frame}[t]{표}

		\begin{block}{표의 caption}
		\end{block}



		\note[item]{}
		\end{frame}


	%	----------------------------------------------------------
	%		tablex
	%	----------------------------------------------------------
		\begin{frame}[t]{표의 배치}

		\begin{block}{표의 배치}
		\begin{itemize}
		\item[h]
		\item[]
		\item[]
		\end{itemize}
		\end{block}



		\note[item]{clear page명령에 의해 표가 커거 강제로 뒤로 배치되는 것을 막을 수 있다.}
		\end{frame}



	%	----------------------------------------------------------
	%		열병합 행병합
	%	----------------------------------------------------------
		\begin{frame}[t]{열병합 행병합}

		\begin{block}{열병합 : 옆으로 병합}
		\begin{itemize}
		\item[]	\textbackslash multi column \{2\} \{c\} \{ 내용 \}
		\end{itemize}
		\end{block}

		\begin{block}{행병합 : 아래로 병합}
		\begin{itemize}
		\item[]	\textbackslash multi row \{2\} \{*\} \{ 내용 \}
		\end{itemize}
		\end{block}


		\note[item]{}
		\end{frame}

	%	----------------------------------------------------------
	%		줄 내부 줄치기
	%	----------------------------------------------------------
		\begin{frame}[t]{표 내부 줄치기}

		\begin{block}{표 내부 줄치기}
		\begin{itemize}
		\item[]	\textbackslash top rule
		\item[]	\textbackslash mid rule
		\item[]	\textbackslash bottom rule
		\end{itemize}
		\end{block}



		\note[item]{}
		\end{frame}



	%	----------------------------------------------------------
	%		tablex
	%	----------------------------------------------------------
		\begin{frame}[t]{table x}

		\note[item]{}
		\end{frame}


	%	----------------------------------------------------------
	%		긴표
	%	----------------------------------------------------------
		\begin{frame}[t]{긴표}

		\begin{block}{긴표}
		\begin{itemize}
		\item[]	\textbackslash begin\{longtable\} \{ |c|c|c|c| \}
		\item[]	\hspace{2em} \textbackslash endfirsthead
		\item[]	\hspace{2em} \textbackslash endhead
		\item[]	\hspace{2em} \textbackslash endfoot
		\item[]	\hspace{2em} \textbackslash endlastfoot
		\item[]	
		\item[]	\textbackslash end\{longtable\} 
		\end{itemize}
		\end{block}



		\note[item]{}
		\end{frame}


	%	----------------------------------------------------------
	%		표속에 각주 넣기
	%	----------------------------------------------------------
		\begin{frame}[t,shrink=0]{표속에 각주 넣기}

			\begin{block} {표속에 각주 넣기}
			\textbackslash footnotemark [ 번호 ] \\
			\textbackslash footnotetext [ 번호 ] \{ 각주 내용 \}
			\end{block}

			\begin{example}
			\begin{itemize}
				\item[]	\textbackslash begin\{table\}[!h]
				\item[]	\textbackslash caption\{페이지 바닥에 ˜각주를 표시하는 표\}
				\item[]	\textbackslash begin\{center\}
				\item[]	\textbackslash begin\{tabular\}\{|c|c|c|\}
				\item[]	\textbackslash hline
				\item[]	GDP \textbackslash footnotemark[1] \&
				\item[]	GDP \textbackslash footnotemark[2] \&
				\item[]	GDP \textbackslash footnotemark[3] \textbackslash \textbackslash
				\item[]	\textbackslash hline
				\item[]	\textbackslash end\{tabular\}
				\item[]	\textbackslash end\{center\}
				\item[]	\textbackslash label\{tab:pagefootnote\}
				\item[]	\textbackslash end\{table\}
				\item[]	
				\item[]	\textbackslash footnotetext[1]\{2007D 한국은행 ‰\}
				\item[]	\textbackslash footnotetext[2]\{2008D 한국은향 추정치 \}
				\item[]	\textbackslash footnotetext[3]\{2008D KDI추정치 \}
			\end{itemize}
			\end{example}

		\note[item]{\textbackslash end \{ table \} 이후에 \textbackslash footnotetext를 위치 시킨다. }
		\note[item]{}
		\note[item]{}
		\note[item]{}
		\note[item]{}
		\note[item]{}

		\end{frame}




	%	==========================================================
	%		그림
	%	----------------------------------------------------------

		\begin{frame}[plain]
		\centering
		\scalebox{10}{그림}

		\note[item]{}
		\end{frame}

	%	----------------------------------------------------------
	%		그림 문서에 넣기
	%	----------------------------------------------------------

		\begin{frame}[t]{그림 : 문서에 그림 넣기}

			\begin{block} {문서에 그림 넣기}
			\begin{itemize}
			\item[]	\textbackslash begin \{ figure \} [ where] 
			\item[]	\textbackslash centering
			\item[]	\textbackslash  caption \{ 그림 설명문 \}
			\item[]	\textbackslash  includegraphics \{ 그림파일명.확장자 \}
			\item[]	\textbackslash  label \{ fig:001 \}
			\item[]	\textbackslash end \{ figure \}
			\end{itemize}
			\end{block}

		\end{frame}



	%	----------------------------------------------------------
	%		include graphics
	%	----------------------------------------------------------

		\begin{frame}[t]{그림}

			\begin{block} {include graphics}
			\begin{itemize}
			\item[]	\textbackslash  includegraphics \{ 그림파일명.확장자 \}
			\item[]	\textbackslash  includegraphics [ scale=0.9 ] \{ 그림파일명.확장자 \}
			\item[]	\textbackslash  includegraphics [ width=1.0\textbackslash textwidth ] \{ 그림파일명.확장자 \}
			\item[]	\textbackslash  includegraphics [ angle=value ] \{ 그림파일명.확장자 \}
			\end{itemize}
			\end{block}

		\end{frame}

	%	----------------------------------------------------------
	%		include pdf
	%	----------------------------------------------------------

		\begin{frame}[t]{그림 : 문서에 pdf 파일 넣기}

			\begin{block} {include pdf}
			\begin{itemize}
			\item[]	\textbackslash includepdf [ pages=-] \{그림파일명.pdf\}
			\item[]	\textbackslash includepdf [ pages=-, fitpaper=true] \{그림파일명.pdf\}
			\item[]	\textbackslash includepdf [ pages=-, scale=0.9] \{그림파일명.pdf\}
			\item[]	\textbackslash includepdf [ pages=-, frame=truee ] \{그림파일명.pdf\}
			\item[]	\textbackslash includepdf [ pages=-, landscape=false ] \{그림파일명.pdf\}
			\end{itemize}
			\end{block}

		\note[item]{ page=- : 전체 페이지를 삽입}
		\note[item]{ fitpaper=trun : 전체 페이지에 배치}

		\end{frame}


	%	----------------------------------------------------------
	%		include path
	%	----------------------------------------------------------

		\begin{frame}[t]{그림 : 그림 파일 경로 지정}

			\begin{block} {include 파일경로}
			\begin{itemize}
			\item[]	\textbackslash includepdf \{./fig/그림파일명.pdf\}
			\item[]	\textbackslash graphicspath\{\{images/\}\}   

			\end{itemize}
			\end{block}

		\note[item]{그래픽 패스는 지정 가능하다}
		\note[item]{pdf 파일의 경로 지정은 ?}

		\end{frame}























	%	==========================================================
	%		수식 
	%	----------------------------------------------------------

		\begin{frame}[plain]
		\centering
		\scalebox{10}{수식}

		\note[item]{}
		\end{frame}




	%	----------------------------------------------------------
	%		수식 사용전 package 선언
	%	----------------------------------------------------------

		\begin{frame}[t]{수식}

			\begin{block} {수식 사용전 package 선언}
			\begin{itemize}
			\item[]	\textbackslash usepackage \{ amsmath \}
			\end{itemize}
			\end{block}

		\note[item]{}
		\end{frame}



	%	----------------------------------------------------------
	%		수식 모드
	%	----------------------------------------------------------
		\begin{frame}[t]{수식 모드}

			\begin{block} {문단 내 배치}
			\end{block}

			\begin{block} {한줄로 배치}
			\end{block}

		\note[item]{}
		\end{frame}


	%	----------------------------------------------------------
	%		수식의 정렬
	%	----------------------------------------------------------
		\begin{frame}[t]{수식의 정렬}

			\begin{block} {왼쪽 정렬}
			\end{block}

			\begin{block} {중간 정렬 : default}
			\end{block}

			\begin{block} {오른쪽 정렬}
			\end{block}

		\note[item]{}
		\end{frame}





	%	----------------------------------------------------------
	%		수식 기본 연산 기호
	%	----------------------------------------------------------
		\begin{frame}[t]{수식 : 기본 연산 기호}

			\begin{block} {수긱 : 기본 연산 기호}
			\begin{description}
			\item[더하기] 	$+$
			\item[빼기] 	$-$
			\item[곱하기] 	$\times$ ( \$ \textbackslash times \$ )
			\item[나누기] 	$\div$ (\$\textbackslash div\$)
			\item[] 		$\frac{1}{2}$ (\$\textbackslash frac\{1\}\{2\}\$)
			\item[] 		$\displaystyle\frac{1}{2}$
						(\$\textbackslash displaystyle\textbackslash frac\{1\}\{2\}\$)
			\end{description}
			\end{block}

		\note[item]{}
		\end{frame}


	%	----------------------------------------------------------
	%		수식 기본 기호
	%	----------------------------------------------------------
		\begin{frame}[t]{기본 기호}

			\begin{block} {기본 기호}
			\begin{description}
			\item[$\therefore$] 	\$ \textbackslash therefore \$
			\end{description}
			\end{block}

		\note[item]{}
		\end{frame}


	%	----------------------------------------------------------
	%		수식에서 한글 입력
	%	----------------------------------------------------------
		\begin{frame}[t]{수식에서 한글 입력}

			\begin{block} {수식에서 한글 입력}
			수식에서 한글 입력기 깨어지는 현상이 무조건 발생한다.\\
			이때는 한글 부분은 \textbackslash \{\}로 묶어서 처리하면 된다.
			\end{block}

		\note[item]{}
		\end{frame}

	%	----------------------------------------------------------
	%		수식 미분
	%	----------------------------------------------------------
		\begin{frame}[t]{미분}

		\note[item]{}
		\end{frame}


	%	----------------------------------------------------------
	%		수식 적분
	%	----------------------------------------------------------
		\begin{frame}[t]{적분}

		\note[item]{}
		\end{frame}


	%	----------------------------------------------------------
	%		수식 행렬
	%	----------------------------------------------------------
		\begin{frame}[t]{행렬}

		\note[item]{}
		\end{frame}







% 	================================================= chapter 	====================
%		box
%	-------------------------------------------------------------------------------

		\begin{frame}[plain]
		\centering
		\scalebox{10}{BOX}

		\note[item]{}
		\end{frame}

	% ------------------------------------------------------------------------------
	%	par box
	% ------------------------------------------------------------------------------

		\begin{frame}[t]{parbox}

			\begin{block} {parbox}
			\textbackslash parbox[position][height][inner-pos]\{width\}\{text\}
			\end{block}

			\begin{block}{position}
			\begin{itemize}
			\item	t --- text is placed at the top of the box.
			\item	c --- text is centred in the box.
			\item	b --- text is placed at the bottom of the box.
			\item	s --- stretch vertically. The text must contain vertically stretchable space for this to work.
			\end{itemize}
			\end{block}


		\note[item]{}
		\end{frame}





	% ------------------------------------------------------------------------------
	%	mbox
	% ------------------------------------------------------------------------------\	\SectionMargin
		\begin{frame}[t]{mbox}


			\begin{block} {mbox}
			\textbackslash mbox\{text\}
			\end{block}

			\begin{example}
			\mbox{mbox mbox mbox mbox mbox mbox mbox}
			\end{example}


		\note[item]{}
 		\end{frame}

	% ------------------------------------------------------------------------------
	%	fbox
	% ------------------------------------------------------------------------------
		\begin{frame}[t]{fbox}


			\begin{block} {f box}
			\textbackslash fbox\{text\}
			\end{block}

			\begin{example}
			\fbox{mbox mbox mbox mbox mbox mbox mbox}
			\end{example}

		\note[item]{}
		\end{frame}

	% ------------------------------------------------------------------------------
	%	pbox
	% ------------------------------------------------------------------------------
		\begin{frame}[t]{pbox}


			\begin{block} {p box}
			\textbackslash pbox[b]\{\textbackslash textwidth\}\{my text\}
			\end{block}

			\begin{example}
			\end{example}


		\note[item]{}
		\end{frame}

	% ------------------------------------------------------------------------------
	%	save box
	% ------------------------------------------------------------------------------
		\begin{frame}[t]{save box}


			\begin{block} {save box}
			\end{block}

		\note[item]{}
		\end{frame}



	% ------------------------------------------------------------------------------
	%	rotate box
	% ------------------------------------------------------------------------------
		\begin{frame}[t]{rotate box}


			\begin{block} {rotate box}
			\end{block}

		\note[item]{}
		\end{frame}


	% ------------------------------------------------------------------------------
	%	colorbox and fcolorbox
	% ------------------------------------------------------------------------------
		\begin{frame}[t]{colorbox and fcolorbox}


			\begin{block} {colorbox}
			\end{block}


			\begin{block} {fcolorbox}
			\end{block}

		\note[item]{}
		\end{frame}



	% ------------------------------------------------------------------------------
	%	resize box
	% ------------------------------------------------------------------------------
		\begin{frame}[t]{resize box}


			\begin{block} {resize box}
			\textbackslash resizebox\{수평 3em\}\{수직 2em\}\{문서 내용 Dunhill style\}

			\end{block}

			\begin{example}
		\resizebox{3em}{2em}{Dunhill style} \\
		\resizebox{4em}{2em}{Dunhill style} \\
		\resizebox{5em}{2em}{Dunhill style} \\
		\resizebox{30em}{2em}{글자 수평 수직 확대} \\
		\resizebox{10em}{3em}{글자 수평 수직 확대} \\

%
%		\resizebox{8em}{1ex}{Dunhill style} \\
%		\resizebox{8em}{2ex}{Dunhill style} \\
%		\resizebox{8em}{3ex}{Dunhill style} \\
%		\resizebox{8em}{4ex}{Dunhill style} \\
%		\resizebox{8em}{5ex}{Dunhill style} \\
%		\resizebox{8em}{6ex}{Dunhill style} \\
			\end{example}


		\note[item]{}
		\end{frame}


	% ------------------------------------------------------------------------------
	%	scale box
	% ------------------------------------------------------------------------------
		\begin{frame}[t]{scale box}


			\begin{block} {scale box}
			\textbackslash scalebox\{스케일 크기 1\}\{문서 내용 화이팅!\}
			\end{block}

			\begin{example}
		\scalebox{1}{화이팅!}\\
		\scalebox{2}{화이팅!}\\
		\scalebox{3}{화이팅!}\\
			\end{example}


		\note[item]{}
		\end{frame}

	% ------------------------------------------------------------------------------
	%	fancybox
	% ------------------------------------------------------------------------------
		\begin{frame}[t]{fancy box}

			\begin{block} {scale box}
				\begin{itemize}
				\item double box
				\item oval box
				\item shadow box
				\end{itemize}
			\end{block}




		\note[item]{}
		\end{frame}



	% ------------------------------------------------------------------------------
	%	makebox
	% ------------------------------------------------------------------------------
		\begin{frame}[t]{makebox}


			\begin{block} {make box}
			\textbackslash makebox [ width ] [ pos ] \{text\}
			\end{block}


			\begin{block} {position}
				\begin{itemize}
				\item	c : center
				\item	l : flushleft
				\item	r : flushright
				\item	s : spread
				\end{itemize}
			\end{block}

			\begin{example}
			c : \makebox[0.4\linewidth][c]	{makebox makebox }\\
			l : \makebox[0.4\linewidth][l]	{makebox makebox }\\
			r : \makebox[0.4\linewidth][r]	{makebox makebox }\\
			s : \makebox[0.4\linewidth][s]	{makebox makebox }\\
			\end{example}


		\note[item]{}
		\end{frame}


	% ------------------------------------------------------------------------------
	%	framebox
	% ------------------------------------------------------------------------------
		\begin{frame}[t]{framebox}



			\begin{block} {frame box}
			\textbackslash framebox [width] [pos] \{text\}
			\end{block}

			\begin{block} {position}
				\begin{description}
				\item	[fboxsep]  the distance between the frame and the content.
				\item	[fboxrule] the thickness of the rule.
				\end{description}
			\end{block}



		\note[item]{}
		\end{frame}




% 	================================================= chapter 	====================
%		mini page
%	-------------------------------------------------------------------------------
		\begin{frame}[t]{minipage}


			\begin{block} {minipage}
			\end{block}

		\note[item]{}
		\end{frame}





	% ------------------------------------------------------------------------------
	%	boxed mini page
	% ------------------------------------------------------------------------------
		
		\begin{frame}[t]{boxedminipage}

			\begin{block} {boxedminipage}
			\end{block}

			\begin{example}
			\end{example}


		\note[item]{}
		\end{frame}


































	%	==========================================================
	%		참고 문헌
	%	----------------------------------------------------------
	%		참고 문헌
	%	----------------------------------------------------------
		\begin{frame}[t]{참고문헌}

		\note[item]{}
		\end{frame}


























































% ------------------------------------------------------------------------------
% End document
% ------------------------------------------------------------------------------
\end{document}


