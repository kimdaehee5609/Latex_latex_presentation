%	-------------------------------------------------------------------------------
%
%
%
%
%
%
%
%
%	-------------------------------------------------------------------------------

%\documentclass[10pt,xcolor=pdftex,dvipsnames,table]{beamer}
\documentclass[10pt,xcolor=pdftex,dvipsnames,table,handout]{beamer}
		
		% 본문 글꼴 크기 : 8,9,10,12,14,17,20

		%	---------------------------------------------------------	
		%	슬라이드 크기 설정 ( 128mm X 96mm )
		%	---------------------------------------------------------	
			\setbeamersize{text margin left=10mm}
			\setbeamersize{text margin right=10mm}

	%	========================================================== 	Package
		\usepackage{kotex}						% 한글 사용
		\usepackage{amssymb,amsfonts,amsmath}	% 수학 수식 사용
		\usepackage{color}					%
		\usepackage{colortbl}					%

		
		%	---------------------------------------------------------	
		%		유인물 출력
		%	---------------------------------------------------------	

			\usepackage{pgfpages}
			\pgfpagesuselayout{2 on 1}[letterpaper]
%			\pgfpagesuselayout{2 on 1}[letterpaper]
%			\pgfpagesuselayout{2 on 1}[letterpaper]
%			\pgfpagesuselayout{2 on 1}[letterpaper]
%			\pgfpagesuselayout{2 on 1}[letterpaper]
%			\pgfpagesuselayout{2 on 1}[letterpaper]
%			\pgfpagesuselayout{2 on 1}[letterpaper]
%			\pgfpagesuselayout{2 on 1}[letterpaper]

	%		========================================================= 	Theme

		%	---------------------------------------------------------	
		%	전체 테마
		%	---------------------------------------------------------	
		%	테마 명명의 관례 : 도시 이름
%			\usetheme{default}			%
%			\usetheme{Madrid}    		%
%			\usetheme{CambridgeUS}    	% -red, no navigation bar
			\usetheme{Antibes}			% -blueish, tree-like navigation bar

		%	----------------- table of contents in sidebar
%			\usetheme{Berkeley}		% -blueish, table of contents in sidebar
									% 개인적으로 마음에 듬

%			\usetheme{Marburg}			% - sidebar on the right
%			\usetheme{Hannover}		% 왼쪽에 마크
%			\usetheme{Berlin}			% - navigation bar in the headline
%			\usetheme{Szeged}			% - navigation bar in the headline, horizontal lines
%			\usetheme{Malmoe}			% - section/subsection in the headline

%			\usetheme{Singapore}
%			\usetheme{Amsterdam}

		%	---------------------------------------------------------	
		%	색 테마
		%	---------------------------------------------------------	
%			\usecolortheme{albatross}	% 바탕 파란
%			\usecolortheme{crane}		% 바탕 흰색
%			\usecolortheme{beetle}		% 바탕 회색
%			\usecolortheme{dove}		% 전체적으로 흰색
%			\usecolortheme{fly}		% 전체적으로 회색
%			\usecolortheme{seagull}	% 휜색
%			\usecolortheme{wolverine}	& 제목이 노란색
%			\usecolortheme{beaver}

		%	---------------------------------------------------------	
		%	Inner Color Theme 			내부 색 테마 ( 블록의 색 )
		%	---------------------------------------------------------	

%			\usecolortheme{rose}		% 흰색
%			\usecolortheme{lily}		% 색 안 칠한다
%			\usecolortheme{orchid} 	% 진하게

		%	---------------------------------------------------------	
		%	Outter Color Theme 		외부 색 테마 ( 머리말, 고리말, 사이드바 )
		%	---------------------------------------------------------	

%			\usecolortheme{whale}		% 진하다
%			\usecolortheme{dolphin}	% 중간
%			\usecolortheme{seahorse}	% 연하다

		%	---------------------------------------------------------	
		%	Font Theme 				폰트 테마
		%	---------------------------------------------------------	
%			\usfonttheme{default}		
			\usefonttheme{serif}			
%			\usefonttheme{structurebold}			
%			\usefonttheme{structureitalicserif}			
%			\usefonttheme{structuresmallcapsserif}			



		%	---------------------------------------------------------	
		%	Inner Theme 				
		%	---------------------------------------------------------	

%			\useinnertheme{default}
			\useinnertheme{circles}		% 원문자			
%			\useinnertheme{rectangles}		% 사각문자			
%			\useinnertheme{rounded}			% 깨어짐
%			\useinnertheme{inmargin}			




		%	---------------------------------------------------------	
		%	이동 단추 삭제
		%	---------------------------------------------------------	
%			\setbeamertemplate{navigation symbols}{}

		%	---------------------------------------------------------	
		%	문서 정보 표시 꼬리말 적용
		%	---------------------------------------------------------	
%			\useoutertheme{infolines}


			
	%	---------------------------------------------------------- 	배경이미지 지정
%			\pgfdeclareimage[width=\paperwidth,height=\paperheight]{bgimage}{./fig/Chrysanthemum.jpg}
%			\setbeamertemplate{background canvas}{\pgfuseimage{bgimage}}

		%	---------------------------------------------------------	
		% 	본문 글꼴색 지정
		%	---------------------------------------------------------	
%			\setbeamercolor{normal text}{fg=purple}
%			\setbeamercolor{normal text}{fg=red!80}	% 숫자는 투명도 표시


		%	---------------------------------------------------------	
		%	itemize 모양 설정
		%	---------------------------------------------------------	
%			\setbeamertemplate{items}[ball]
%			\setbeamertemplate{items}[circle]
%			\setbeamertemplate{items}[rectangle]


		%	---------------------------------------------------------	
		%	상자 모양새 설정
		%	---------------------------------------------------------	
%			\setbeamertemplate{blocks}[rounded,shadow=true]
%			\begin{block}
%			\begin{theorem}
%			\begin{lemma}
%			\begin{proof}
%			\begin{corollary}
%			\begin{example}
%			\begin{exampleblock}
%			\begin{alertblock}




		\setbeamercovered{dynamic}






% ------------------------------------------------------------------------------
% Begin document (Content goes below)
% ------------------------------------------------------------------------------
	\begin{document}
	

			\title{LATEX}
			\subtitle{사용설명서}
			\author{김대희}
			\date[2015.06.30]{2015년 6월}
			\institute[KTS]{(주)서영엔지니어링 \texttt{http://symsone.seoyeong.co.kr/}}



	%	==========================================================
	%
	%	----------------------------------------------------------
		\begin{frame}[plain]
		\titlepage
		\end{frame}

		\begin{frame}[plain]
		\end{frame}


	%	==========================================================
	%		문서 클래스
	%	----------------------------------------------------------

		\begin{frame}[plain]
		\Huge{문서 클래스}
		\end{frame}



	%	----------------------------------------------------------
	%		문서 클래스
	%	----------------------------------------------------------
		\begin{frame}[t,shrink=20]{문서 클래스}


			\begin{block} {장, 절의 설정}
			\begin{description}[1234567890]
			\item [article] 과학 학술지, 프리젠테이션, 짧은 보고서, 프로그램 문서, 초대장 등에 쓰이는 아티클용 클래스
			\item [proc article] 클래스에 기초한 프로시딩을 위한 클래스 
			\item [minimal] 최소 문서 양식 클래스. 페이지 크기와 기본 글꼴만을 설정한다. 주로 디버깅을 위하여 사용함.
			\item [report] 여러 장(chapter)으로 이루어진 긴 보고서, 작은 책, 박사학위 논문 등에 쓰이는 클래스.
			\item [book] 진짜 책을 만들기 위한 클래스.
			\item [slides] 슬라이드 제작용 클래스. 큰 산세리프 글꼴을 사용한다. 이것 대신 FoilTEX의 사용도 고려해볼 수 있다.a
			\end{description}
			\end{block}


		\end{frame}


	%	----------------------------------------------------------
	%		문서 클래스 옵션
	%	----------------------------------------------------------
		\begin{frame}[t,shrink=20]{문서 클래스}


			\begin{block} {문서 클래스 옵션}
			\begin{description}[1234567890]
			\item [10pt] 11pt, 12pt 문서 기본 글꼴 크기를 설정한다.
			\item [letterpaper] a4paper,a5paper, b5paper, executivepaper,legalpaper
			\item [fleqn] 수식을 가운데 정렬이 아닌 왼쪽 정렬로 식자한다.
			\item [leqno] 수식 번호를 수식의 오른쪽이 아닌 왼쪽에 표시되도록 한다.
			\item [titlepage] notitlepage 표지 뒤에 새로운 페이지를 시작하도록 할 것인지 지정한다. report와 book은 새 페이지를 만든다.
			\item [onecolumn] twocolumn 문서를 1단 또는 2단으로 조판하도록 지시한다.
			\item [twoside] oneside 양면인쇄용 출력물 생성.\\ 단면(article,report) 양면 (book)
			\item [landscape] 레이아웃을 가로가 긴 형식(landscape)으로 변경한다.
			\item [openright] openany 새로운 장을 홀수쪽에서 시작.\\
								book 클래스에서는 홀수쪽에서 시작하는 것이 기본값이다.
			\end{description}
			\end{block}

		\end{frame}



	%	==========================================================
	%		패키지
	%	----------------------------------------------------------
		\begin{frame}[t]{패키지}
		\end{frame}



	%	==========================================================
	%		쪽 양식
	%	----------------------------------------------------------
		\begin{frame}[t]{쪽양식}

			\begin{block} {쪽양식}
			\begin{description}[1234567890]
			\item [plain] 쪽 번호를 쪽의 아래쪽 바닥글에 중앙정렬하여 찍는다. 쪽 양식의 기본값이다.
			\item [headings] 현재 장 표제와 쪽 번호를 각 쪽의 머리글에 적는다. 바닥글은 비운다.
			\item [empty] 머리글과 바닥글을 모두 비운다.
			\end{description}
			\end{block}

		\end{frame}


	%	==========================================================
	%		장, 절의 설정
	%	----------------------------------------------------------
		\begin{frame}[t]{장, 절의 설정}

			\begin{block} {장, 절의 설정}
			\begin{enumerate}
			\item	\textbackslash part
			\item	\textbackslash chapter
			\item	\textbackslash section
			\item	\textbackslash sub section
			\item	\textbackslash sub sub section
			\item	\textbackslash paragraph
			\item	\textbackslash sub paragraph
			\end{enumerate}
			\end{block}

		\end{frame}




	%	==========================================================
	%		title page
	%	----------------------------------------------------------
		\begin{frame}[t]{표지 작성}


			\begin{block} {표지작성}
			\begin{description}[1234567890]
			\item [\textbackslash title] 	문서 제목
			\item [\textbackslash author]	문서 저자
			\item [\textbackslash date] 	작성일
			\item [\textbackslash maketitle] 타이틀 표시
			\end{description}
			\end{block}

		\end{frame}




	%	==========================================================
	%		초록 작성
	%	----------------------------------------------------------
		\begin{frame}[t]{초록 작성}

			\begin{block} {초록 작성}
			\begin{description}[12345678901234567]
			\item [\textbackslash begin\{abstract\}] 	문서 제목
			\item [초록 내용]						초록 내용
			\item [\textbackslash end\{abstract\}] 	작성일
			\end{description}
			\end{block}


		\end{frame}




	%	==========================================================
	%		목차 작성
	%	----------------------------------------------------------
		\begin{frame}[t]{목차 작성}

			\begin{block} {목차 작성}
			\begin{description}[12345678901234567]
			\item [\textbackslash table of contents] 	문서 내용 목차
			\item [\textbackslash list of figures]		그림 목차
			\item [\textbackslash list of tables] 		표 목차
			\end{description}
			\end{block}



		\end{frame}



	%	==========================================================
	%		주석문 처리
	%	----------------------------------------------------------

		\begin{frame}[t]{주석문 처리}

			\begin{block} {주석문 처리}
			\textbackslash \%
			\end{block}

		\end{frame}


	%	----------------------------------------------------------
	%		각주
	%	----------------------------------------------------------

		\begin{frame}[t]{각주}

			\begin{block} {각주}
			\textbackslash footnote \{ 각주 내용 \}
			\end{block}


		\end{frame}



	%	----------------------------------------------------------
 	%		난외주
	%	----------------------------------------------------------

		\begin{frame}[t]{난외주}

			\begin{block} {난외주}
			\textbackslash marginpar \{ 난외주 내용 \}
			\end{block}

		\end{frame}


	%	----------------------------------------------------------
	%		인용문
	%	----------------------------------------------------------

		\begin{frame}[t]{인용문}

			\begin{block} {인용문}
			\textbackslash begin \{ quote \}\\
			~인용문 내용 \\
			\textbackslash end \{ quote \}
			\end{block}

		\end{frame}




	%	----------------------------------------------------------
	%		 상호 참조
	%	----------------------------------------------------------

		\begin{frame}[t]{상호참조}

			\begin{block} {상호참조}
				\begin{itemize}
				\item \textbackslash label \{참조 기호 \}
				\item \textbackslash ref \{참조할 기호 \}
				\item \textbackslash pageref \{참조할 기호 \}
				\end{itemize}
			\end{block}

		\end{frame}



	%	----------------------------------------------------------
	%		counter
	%	----------------------------------------------------------
		\begin{frame}[t]{counter}

			\begin{block} {\textbackslash setcounter \{ tocdepth \} }
			\end{block}

			\begin{block} {\textbackslash setcounter \{ secnumdepth \} \{ n \} }
			\end{block}


			\begin{center}
			\begin{table}
			\begin{tabular}{ l l }
				\hline
				부(part)				&-1\\
				장(chapter)			&0\\
				절(section)			&1\\
				소절(subsection)		&2\\
				소소절(subsubsection)	&3\\
				문단(paragraph)		&4\\
				소문단(subparagraph)	&5\\
				\hline
			\end{tabular}
			\end{table}
			\end{center}











		\end{frame}

	%	----------------------------------------------------------
		\begin{frame}[plain]
		\end{frame}

	%	==========================================================
	%		문단 모양
	%	----------------------------------------------------------

		\begin{frame}[plain]
		\Huge{문단}
		\end{frame}


	%	----------------------------------------------------------
	%		Line and page Breaking
	%	----------------------------------------------------------
		\begin{frame}[t]{Line and page Breaking}

			\begin{block} {Line and page Breaking}
			\begin{enumerate}
			\item	\textbackslash clear page
			\item	\textbackslash clear double page
			\item	\textbackslash hyphenation
			\item	\textbackslash line break
			\item	\textbackslash new line
			\item	\textbackslash no line break
			\item	\textbackslash no page break
			\item	\textbackslash page break
			\end{enumerate}
			\end{block}

		\end{frame}

	%	----------------------------------------------------------
	%		문단 첫줄 들여쓰기
	%	----------------------------------------------------------
		\begin{frame}[t]{문단 첫줄 들여쓰기}

			\begin{block} {문단 첫줄 들여쓰기}
			\textbackslash par indent
			\end{block}

			\begin{example}
			\textbackslash setlength \{ \textbackslash parindent \} \{ 0.0cm \}\\
			\end{example}

		\end{frame}

	%	----------------------------------------------------------
	%		문단과 문단 사이의 간격
	%	----------------------------------------------------------
		\begin{frame}[t]{문단과 문단 사이의 간격}

			\begin{block} {문단과 문단 사이의 간격}
			\textbackslash par skip
			\end{block}

			\begin{example}
			\textbackslash setlength \{ \textbackslash parskip \} \{ 0.0pt \}\\
			\textbackslash setlength \{\textbackslash parskip \} \{ 1cm plus 4mm minus\}	\\		
			\end{example}

			\begin{example}
			\textbackslash small skip
			\textbackslash med skip
			\textbackslash big skip
			\end{example}

		\end{frame}


	%	----------------------------------------------------------
	%		문단내에서의 줄간격
	%	----------------------------------------------------------
		\begin{frame}[t,allowframebreaks]{문단내에서의 줄간격}

			\begin{block} {문단내에서의 줄간격}
			\textbackslash usepackage \{ setspace \} \\
			\hfill \textbackslash singlespacing \\
			\hfill \textbackslash onehalfspacing \\
			\hfill \textbackslash doublespacing \\
			\hfill \textbackslash setstretch \{ $<$ $>$ \} \\
			\hfill \textbackslash linespread \{ $<$ factor $>$ \}
			\end{block}

			\begin{example}
			\textbackslash linespread \{ 1.6 \} : double-spacing \\
			\textbackslash linespread \{ 1.3 \} : one-and-a half spacing 
			\end{example}
	
			\newpage
			\begin{example}
			\textbackslash begin \{ doublespace \} \\
			.............. \\
			\textbackslash end \{ doublespace \} \\
			\end{example}

			\begin{example}
			\textbackslash begin \{ spacing \} \{ 2.0 \} \\
			.............. \\
			\textbackslash end \{ spacing \} \\
			\end{example}

		\end{frame}



	%	----------------------------------------------------------
	%		문자간의 간격 띄우기
	%	----------------------------------------------------------
		\begin{frame}[t,allowframebreaks]{문자간의 간격 띄우기}

			\begin{block} {문자간의 간격 띄우기}
			\begin{enumerate}

			\item $\sim$\\
			\item \textbackslash hspace \{ 2cm \} \\
			\item \textbackslash quad 	(여백입력)\\
			\item \textbackslash qquad 	(여백입력 두배크기)\\
			\item \textbackslash textspace \\
			\item \textbackslash large space \\
			\item \textbackslash medium space \\
			\item \textbackslash small space \\
			\item \textbackslash negative space \\
			\end{enumerate}

			\end{block}
		\end{frame}


	%	----------------------------------------------------------
	%		미리 정의된 문자열
	%	----------------------------------------------------------
		\begin{frame}[t,allowframebreaks]{미리 정의된 문자열}

			\begin{center}
			\rowcolors{1}{blue!10}{yellow!10}
			\begin{table}
			\begin{tabular}{ l l l  }
				명령어	&사용예	&설명\\
				\hline
				\textbackslash today 	&\today 		&현재 사용 언어에서의 현재 날짜 표기\\
				\textbackslash TeX 	&\TeX 		&최고의 조판 시스템의 이름\\
				\textbackslash LaTeX 	&\LaTeX 		&지금 우리가 배우고 있는 것의 이름\\
				\textbackslash LaTeXe 	&\LaTeXe		&LATEX의 최신판\\
			\end{tabular}
			\end{table}
			\end{center}

		\end{frame}


	%	----------------------------------------------------------
	%		특수문자
	%	----------------------------------------------------------
		\begin{frame}[t,allowframebreaks]{특수문자}

			\begin{columns}[t]
			\begin{column}{.4\textwidth}
			\begin{block} {특수문자}
			\begin{enumerate}
			\item \# : \textbackslash \# \\
			\item \$ : \textbackslash \$ \\
			\item \% : \textbackslash \% \\
			\item \& : \textbackslash \& \\
			\item \_ : \textbackslash \_ \\
			\item \{ : \textbackslash \{ \\
			\item \} : \textbackslash \} \\
			\item \textbackslash  : \textbackslash textbackslash
			\end{enumerate}
			\end{block}
			\end{column}

			\begin{column}{.4\textwidth}
			\begin{block} {특수문자}
			\begin{enumerate}
			\item \_ : \textbackslash \_ \\
			\item \^{} : \textbackslash \^{} \\
			\item $\sim$ : \$ \textbackslash sim \$
			\end{enumerate}
			\end{block}
			\end{column}
			\end{columns}

			\newpage
			\begin{columns}[t]
			\begin{column}{.4\textwidth}
			\begin{block} {특수문자}
			\begin{enumerate}
			\item $\cdot$ 		: \$ $\cdot$ \$ \\
			\item $\circ$ 		: \$ $\circ$ \$ \\
			\item $\bullet$ 		: \$ $\bullet$ \$ \\
			\item $\centerdot$		: \$ $\centerdot$ \$ \\
			\item $\div$ 			: \$ $\div$ \$ \\
			\item $\times$ 		: \$ $\times$ \$ \\
			\end{enumerate}
			\end{block}
			\end{column}

			\begin{column}{.4\textwidth}
			\begin{block} {특수문자}
%			\begin{enumerate}
%			\item \approx		:  	\\
%			\item \neq		: 	\\
%			\item \infty		: 	\\
%			\end{enumerate}
			\end{block}
			\end{column}
			\end{columns}


			\textcircled{\small}\\

		\end{frame}





	%	----------------------------------------------------------
	%		선 그리기
	%	----------------------------------------------------------
		\begin{frame}[t]{선 그리기}

			\begin{block} {선 그리기}
			\textbackslash rule \{ \textbackslash linewidth \} \{ 두께 \}
			\end{block}

			\begin{example}
			\textbackslash rule \{ 4cm \} \{ 2mm \}
			\end{example}

			\rule{4cm}{2mm}
		\end{frame}



	%	----------------------------------------------------------
		\begin{frame}[plain]
		\end{frame}

	%	==========================================================
	%		글꼴 모양
	%	----------------------------------------------------------

		\begin{frame}[plain]
		\Huge{글꼴 모양}
		\end{frame}



	%	----------------------------------------------------------
	%		글꼴 모양
	%	----------------------------------------------------------


		\begin{frame}[t,allowframebreaks]{글꼴 모양}

			\begin{table}
			\begin{tabular}{ l l l  }
				명령어	&환경	&결과\\
				\hline
				\textbackslash textnarmal 	&textnarmal 	&\textnormal{결과}\\
				\textbackslash textit 		&itshape		&\textit{결과}\\
				\textbackslash emph 		&없음		&\emph{결과}\\
				\textbackslash textbf 		&bfseries		&\textbf{결과}\\
				\textbackslash underline 	&밑줄		&\underline{결과}\\
			\end{tabular}
			\end{table}

		\end{frame}




	%	==========================================================
	%		글꼴 크기
	%	----------------------------------------------------------



		\begin{frame}[t,allowframebreaks]{글꼴 크기}

			\begin{table}
			\begin{tabular}{ l l l  }
				명령어		&결과\\
				\hline
				\textbackslash tiny			&\tiny{결과}\\
				\textbackslash scriptsize		&\scriptsize{결과}\\
				\textbackslash footnotesize		&\footnotesize{결과}\\
				\textbackslash normalsize		&\normalsize{결과}\\
				\textbackslash large			&\large{결과}\\
				\textbackslash Large			&\Large{결과}\\
				\textbackslash LARGE			&\LARGE{결과}\\
				\textbackslash huge			&\huge{결과}\\
				\textbackslash Huge			&\Huge{결과}\\
			\end{tabular}
			\end{table}

		\end{frame}









	%	----------------------------------------------------------
		\begin{frame}[plain]
		\end{frame}
	%	==========================================================
	%		개조식 문서
	%	----------------------------------------------------------

		\begin{frame}[plain]
		\Huge{개조식 문서}
		\end{frame}


	%	----------------------------------------------------------
	%		List
	%	----------------------------------------------------------
		\begin{frame}[t]{List}

		\end{frame}


	%	----------------------------------------------------------
	%		itemize
	%	----------------------------------------------------------
		\begin{frame}[t]{itemize}

		\end{frame}


	%	----------------------------------------------------------
	%		itemize 기호 모양 바꾸기
	%	----------------------------------------------------------
		\begin{frame}[t]{itemize 기호 모양 바꾸기}

		\end{frame}


	%	----------------------------------------------------------
	%		enumerate
	%	----------------------------------------------------------
		\begin{frame}[t]{enumerate}

		\end{frame}

	%	----------------------------------------------------------
	%		enumerate 기호 모양 바꾸기
	%	----------------------------------------------------------
		\begin{frame}[t]{enumerate 기호 모양 바꾸기}

		\end{frame}


	%	----------------------------------------------------------
	%		description
	%	----------------------------------------------------------
		\begin{frame}[t]{description}

		\end{frame}





	%	----------------------------------------------------------
		\begin{frame}[plain]
		\end{frame}
	%	==========================================================
	%		표 그리기
	%	----------------------------------------------------------

		\begin{frame}[plain]
		\Huge{표 그리기}
		\end{frame}


	%	----------------------------------------------------------
	%		표
	%	----------------------------------------------------------
		\begin{frame}[t]{표}

		\end{frame}


	%	----------------------------------------------------------
	%		tablex
	%	----------------------------------------------------------
		\begin{frame}[t]{tablex}

		\end{frame}


	%	----------------------------------------------------------
	%		열병합표
	%	----------------------------------------------------------
		\begin{frame}[t]{열병합표}

		\end{frame}

	%	----------------------------------------------------------
	%		행병합표
	%	----------------------------------------------------------
		\begin{frame}[t]{행볍합표}

		\end{frame}


	%	----------------------------------------------------------
	%		긴표
	%	----------------------------------------------------------
		\begin{frame}[t]{긴표}

		\end{frame}


	%	----------------------------------------------------------
	%		표속에 각주 넣기
	%	----------------------------------------------------------
		\begin{frame}[t,shrink=40]{표속에 각주 넣기}

			\begin{block} {표속에 각주 넣기}
			\textbackslash footnotemark [ 번호 ] \\
			\textbackslash footnotetext [ 번호 ] \{rkrwn sodyd \}
			\end{block}

			\begin{example}
			\begin{itemize}
				\item[]	\textbackslash begin\{table\}[!h]
				\item[]	\textbackslash caption\{페이지 바닥에 ˜각주를 표시하는 표\}
				\item[]	\textbackslash begin\{center\}
				\item[]	\textbackslash begin\{tabular\}\{|c|c|c|\}
				\item[]	\textbackslash hline
				\item[]	GDP \textbackslash footnotemark[1] \&
				\item[]	GDP \textbackslash footnotemark[2] \&
				\item[]	GDP \textbackslash footnotemark[3] \textbackslash \textbackslash
				\item[]	\textbackslash hline
				\item[]	\textbackslash end\{tabular\}
				\item[]	\textbackslash end\{center\}
				\item[]	\textbackslash label\{tab:pagefootnote\}
				\item[]	\textbackslash end\{table\}
				\item[]	
				\item[]	\textbackslash footnotetext[1]\{2007D 한국은행 ‰\}
				\item[]	\textbackslash footnotetext[2]\{2008D 한국은향 추정치 \}
				\item[]	\textbackslash footnotetext[3]\{2008D KDI추정치 \}
			\end{itemize}
			\end{example}
		\end{frame}




	%	----------------------------------------------------------
		\begin{frame}[plain]
		\end{frame}
	%	==========================================================
	%		그림
	%	----------------------------------------------------------
		\begin{frame}[plain]
		\Huge{그림}
		\end{frame}


	%	----------------------------------------------------------
	%		그림
	%	----------------------------------------------------------
		\begin{frame}[t]{그림}

		\end{frame}



	%	==========================================================
	%		수식 
	%	----------------------------------------------------------

		\begin{frame}[plain]
		\Huge{수식}
		\end{frame}


	%	----------------------------------------------------------
	%		수식 모드
	%	----------------------------------------------------------
		\begin{frame}[t]{수식 모드}

		\end{frame}


	%	----------------------------------------------------------
	%		수식의 정렬
	%	----------------------------------------------------------
		\begin{frame}[t]{수식의 정렬}

		\end{frame}





	%	----------------------------------------------------------
	%		수식 기본 연산 기호
	%	----------------------------------------------------------
		\begin{frame}[t]{기본 연산 기호}

		\end{frame}


	%	----------------------------------------------------------
	%		수식 미분
	%	----------------------------------------------------------
		\begin{frame}[t]{미분}

		\end{frame}


	%	----------------------------------------------------------
	%		수식 적분
	%	----------------------------------------------------------
		\begin{frame}[t]{적분}

		\end{frame}


	%	----------------------------------------------------------
	%		수식 행렬
	%	----------------------------------------------------------
		\begin{frame}[t]{행렬}

		\end{frame}




	%	----------------------------------------------------------
		\begin{frame}[plain]
		\end{frame}
	%	==========================================================
	%		참고 문헌
	%	----------------------------------------------------------

		\begin{frame}[plain]
		\Huge{참고 문헌}
		\end{frame}



	%	----------------------------------------------------------
	%		참고 문헌
	%	----------------------------------------------------------
		\begin{frame}[t]{참고문헌}

		\end{frame}



























% ------------------------------------------------------------------------------
% End document
% ------------------------------------------------------------------------------
\end{document}


